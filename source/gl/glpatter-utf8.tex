% This is the file glpatter.tex, version 2.3a
% It is the source file for the Galician patterns
% 
% (c) Javier A. Múgica; 2006, 2007, 2008
% License: LPPL version 1.3
% 
% LPPL maintenance status: maintained
% Current Maintainer: Javier A. Múgica
%
%
% The patterns follows approximately the scheme developed by Javier Bezos for esphyph.tex
% In order to generate the patterns just type
%     tex --ini --8bit glpatter-utf8.tex
%
%	For bug reports and comments:
%
%	Javier Múgica,   javier at digi21.eu
\input mkpatter.tex
\filename{hyph-utf8-gl.tex}
\tracingexceptions=1

%The following is just a definition that will not be used. (See 25  lines below)
\begin{code}
	\def\Ti{\encodingreplacements{á é í ó ú ñ ü ï}{^^e1 ^^e9 ^^ed ^^f3 ^^fa ^^f1 ^^fc ^^ef}}
\end{code}

&Va={a e o}
&Vp={i u}
&Vi={&Va u}
&Va+={&Va á}
&V={a e i o u á é í ó ú}
&Vv={a e i}
&Vai={a i}
&Vvac={á é í}
&L={l r}
%ñ non se inclúe nos seguintes grupos
&C-={b c d f g h k m n p q s t v w x z}	%Consoantes excepto as líquidas
&C={&C- l r}
&Cl={b c d f g k p t v}	%As que van unidas á líquida, agás r e l
&C!l={m n q s x y z h}%As que van separadas da líquida, agás l e r
&G={b c d l m n r s t x}%Consoante inicial de certos grupos pouco frecuentes

\templatedef{prefixos}{#1&{V}2}
\templatedef{3a_ppi}{#2u3 \newline}	%3a. persoa do perfecto de indicativo, rematadas en -ou, -eu, -iu.
													%O confilcto cun prefixo máis habitual é na primeira conxugación.
%
%Uncomment the %\Ti line if you want T1 paterns, and the %\letters line if  you want those characters
%to be catcoded as letters and the correspoding caplital ones to be assigned the right \lcode.
%In that case you will need an utf-8 engine to process this file (luatex or xetex) or, better, use the
%single-byte encoded version of this fie: glpatter.tex.
%
%\Ti
%\letters{á é í ó ú ñ ü ï}

%The commented lines that follow will be placed in the generated file,
%and do not apply to this file.
\expandafter\let\expandafter\lee\csname ~~lee\endcsname
\keepcomments
% This is the file hyph-gl.tex, version 2.3a-utf8
% Hyphenation patterns for Galician, written in the utf8 encoding.
%
% Generated with the mkpattern utility (v. 1.1), on 2008/06/13.
% The original source file were glpatter-utf8.tex
% This is a generated file
%
% (c) Javier A. Múgica; 2006, 2007, 2008
% License: LPPL version 1.3
% 
% LPPL maintenance status: maintained
% Current Maintainer: Javier A. Múgica
%
%	For bug reports and comments:
%
%	Javier Múgica,   javier at digi21.net
\let\lee\relax
\halfcomments
\iwrite{%}
\iwrite{% Note that there is no 'j' nor 'y' in Galician}
\nocomments

\begin{pseudopatterns}

3ï3a \newline
a1a e1e o1o zo2o a1á e1é o1ó 2&{Va}&{Vp}.
&{Vi}1i&{Va+} 2i&{Va+}.

\newline
\exceptions{2ch 2gh 2kh}{}
\exceptions{t2l 2tl.}{}
1ñ
1&{C}
2&{C}&{C-}
c4h 2ch. 2g2h 2k2h \newline
\newline
&{Cl}2&L 2&{Cl}&L.
2&{C!l}&L

2t2l \newline
2lr l4l 2ll.
2rl r2r 2rr.

\template{2&G3#}
p2t c2t c2n p2s m2n g2n f2t p2n c2z t2s
\end{template}
san4c5t plan4c5t

\template{4#.}
pt ct cn ps mn gn ft pn cz ts
\end{template}

un4ha 2non. 3mente. o2hib alde2h \newline
\newline
%palabras malsonates
4caca4 4cago4 4caga4 4cagas. 4puta4 4puto4 4meo. 4mea. \newline
4meable. 4meables. 4peido4

\exceptions{.pre1á2 .su2b1i2 .su2b1í2}{}	%If duplicate patterns were simply added...
\exceptions{.pre2á}{.pre2á2}							%This is error prone. I think that writting
\exceptions{.sub2i .sub2í}{.su2b2i2,.su2b2í2}%all the exceptions here together is the best that can be done.
\begin{prefixos}	%Prefixos
acto afro aero anfi anglo .ante .anti .arqui auto
biblio bio cardio cefalo ciclo cito contra cripto cromo crono
deca .deza dinamo ecano eco electro endo ento entre
euco euro extra fono foto franco gastro xeo
hecto helio hemato hemi hexa hidro hipe2r histo homeo homo
ibero icono .in .indo infra .inte2r intra .iso kilo
macro magneto maxi mega megalo melano micro mili mini
multi miria mono .nano necro .neo norte octo octa
omni paleo para penta piezo pluri poli .pos2t .pre
.pro proto pseudo radio retro sobre semi
socio .su2b super supra .tele termo tetra topo
.tri tropo ultra xeno
\end{prefixos}

\newline
ti2o3qu ti2o3co
bi1u2ní
o2i3de o2i3dal
deca2e3ment
2al. 2a2is.
pe3r2e3mia
hiper3r \newline
famili2a familia3r
mini2a3tur
para2u3gas
paraí3so

\begin{3a_ppi}
\put{%Terceiras persoas do pasado perfecto}\newline
libero atopo enaxeno
\end{3a_ppi}

\newline
2os. 2o3so. 2o3sos. 2o3sa. 2o3sas. 2o3samente. \newline
2i3co. 2i3cos. 2i3ca. 2i3cas. \newline

\newline
\put{%O prefixo co}
\exceptions{.co1á2}{}
.co1&V2

%Pode non ser prefixo (expecto os derivados de coar)
\template{.co2# \newline}
\put{%Formas do verbo coar con pronomes enclíticos}\newline ar á2 abá
acerv andro ano añar año art etan enci erci inci ira iro ita
\end{template}

%Non é prefixo (os que non son da forma &V3 non necesitan ir un por un)
\template{co2# \newline}	%Non é necesario coia, coiote
a3gul á3gul a3la. a3las. a3lescen a3lición. a3licions. a3na. a3nas.
antriñ a3ñadeir a3tí. a3tís. e3ficien e3lernos. e3llo. e3lla. e3llos. e3llas.
e3lleir enll enxía e3sita. e3sitas. e3táne e3vo. e3va. e3vos. e3vas.
i3dado iei imbra imbrá intreau. í3ña i3ña i3ñei i3pú. i3pús.
i3ra. i3ras. i3raza i3ro. i3ros. i3ta. i3tas. i3tado. i3to. i3tos.
i3tel i3tío. i3tus. u3c u3lomb u3try u3qui u3rel
u3sa. u3sas. u3so. u3sos. u3selo u3tad u3to. u3tos. u3vini u3z
\end{template}

\newline
\put{%O prefixo des}
.de2s1&{V}2
3se. 3s2es. 3sa. 3s2as. de3s2outr 3s2emos. 3s2edes. 3s2en.

\exceptions{de3s2esper}{de3s2esper de4s3esperanz}
\template{de3s2# \newline}	%Prefixos .de (máis ou menos)
a3crali a3guisa a3lini a3ngr a3ñ a3rrollis astr a3zo e3c e3que e3guid e3la
ensib e3ñ ert ért esper e3pér e3x é3x i3der ign ígn i3nenc ingr iste isti o3lac o3lad old
o3lidari uetud sulf
\end{template}

\template{.des2# \newline}  %Prefixos dubidosos ou afastados da orixe etimolóxica. Non se inclúen os que por ser moi breves poderían coincidir con prefixos .des formados ad hoc, xa que impedirían a división de palabras comúns (desecar, deselar).
abor afia afía afío air emboc embóc empeñ empéñ enlac enlaz enlác enláz
envol envól idia ora
\end{template}

\keepcomments

\newline
%Excepcións ó prefixo in
\template{.in2# \newline}
a3misib. a3mov a3ne. a3nic a3nid á3nime antes. au e3dia é3dit e3fab e3narr 
epc ept erc ert erm erv e3siv e3xora i3ci i3cu i3mig i3miza i3qui
o3cen o3cui o3cuo o3cul ó3cul o3pia. o3sili o3sit o3tróp o3trop uit u3lase u3lina 
unda u3sita ú3til
\end{template}
in2o4cular \newline

\newline
%Excepcións ó prefixo inter
.inter3r
\template{.inte3r2#}
és. e3sa é3sa e3sá e3so é3so e3só ior i3no. i3nos. i3na. i3nas. i3nid
\end{template}

\newline
%Os prefixos mal e ben
\exceptions{.be2n1e2 .be2n1é2 .ma2l1e2 .ma2l1é2}{.be2ne2,      ,.ma2le2,      }

\halfcomments
.be2n1&{V}2
\template{be3n2# \newline}	%Non é prefixo (ou non se nota)
ign i3merí i3nés i3nes i3toíta 
\end{template}

\newline
.ma2l1&{V}2
\template{.mal2# \newline}	%Pode non ser prefixo
abar abár aco acó armad ogr ura axa
\end{template}
\template{ma3l2# \newline}	%Non é prefixo. Non é neceario malos, malas
a3cía  a3citan a3gueñ aio aia andrín andrin a3quita ar. a3res. a3ria.
a3to. a3tos. a3ta. a3tas. a3tión aui eabl eabil eico eolar e3ta. e3tas. e3tín e3teiro
eza ia. ian i3cia i3cios ign i3kita inke ó3fago ó3nic o3nato o3nilue
\end{template}
ma4l3ianq
\template{.mal1# \newline} %É prefixo e comenza por e
educ encar ensin entend
\end{template}

\newline
\put{%Excepcións ó prefixo poli}
\template{poli2# \newline}
u3r o3me arq árq éste andr antea
\end{template}
expoli2

\newline
\put{%Excepcións ó prefixo post}
\template{pos3t2# \newline}
a. as. al. ais. a3llo e. es. ear. e3la. e3las. er. erg e3rid e3rior i3go i3la
illón ín. i3te. i3zo. i3zos. i3za. i3zas. os. oiro ó3ni u3la u3lo u3le u3ra. u3ras.
\end{template}

\newline
\put{%Excepcións ó prefixo pre}
\template{.pre2# \newline} %Poden non ser prefixos
amar \put{%Formas do verbo prear con pronomes enclíticos}\newline ar á abá
\end{template}
\template{pre2# \newline} %Non son prefixos
as. a3da. a3das. á2 abá i3t o3cup o3cúp
\end{template}

\newline
\put{%Excepcións ó prefixo pro}
\template{pro2# \newline} %Non son prefixos. Non é necesario proer
e3za í3do ust
\end{template}

\newline
\exceptions{.re2e2 .re2é2}{.ree2,.reé2}
\put{%O prefixo re} %Faltan excepcións
.re2&{V}2

 
\newline
\put{%O prefixo sub: l, r e excepcións}\newline
.su2b3l .su2b3r

\template{.sub2# \newline} %Poden non ser prefixos
\put{%Formas do verbo subir con pronomes enclíticos}\newline i í
eriz orna
\end{template}
\newline
sub3índic sub3indic sub3indiz

\template{.sub4# \newline}
lev lim
\end{template}

\template{su3b2# \newline} %Non son prefixos
e3la. e3las. é3rico e3rina. e3rinas. eroso iote ulado orno. ornos. urbio
\end{template}
\template{su3b4# \newline}
liminar repción reptici
\end{template}

\newline
\put{%Excepcións ó prefixo tri}
\template{tri2# \newline} %Non son prefixos (ou non se nota)
a3ga. a3gas.  al. a3les. angul á3sico. estin %(De Trieste)
unf unvir
\end{template}

\newline
\put{%Excepcións a varios prefixos que son terminacións verbais}
%As combinacións con pronomes enclíticos (son centos)
%deberanse evitar nos patróns de cada prefixo cando sexa necesario
%Inclúense as combinacións máis habituais
\template{2&{Vai}#.}
3do 3da 3dos 3das ndo r %inf. xer. part.
\end{template}

\template{2&{Vv}#.}
3res rmos rdes 3ren \newline %inf. conx.
rme rte rlle rnos rvos rlles %inf. con pron. indir.
\end{template}

\template{2&{Vv}3#.}
dor dora dores doiro doiros doira doiras deiro deiros deira deiras %subst. deriv.
lo los la las %inf. con pron. dir.
rei rás rá remos redes rán \newline %futuro
ría rías ri1amos ri1ades rían \newline %cond.
de %imp.
\end{template}
\template{2&{Vvac}3#.}
deo dea deos deas %imp. con pron. dir.
\end{template}

\template{2#.}
%coar, inar, entrenar, prear, superar
as a3mos a3des an \newline
a3ba a3bas a3bamos a3bades a3ban \newline
a3ches astes a3ron \newline
es e3mos e3des en \newline
a3se a3ses á3semos á3sedes a3sen \newline
\end{template}
%pero permitimos a ruptura para coar, prear, voar, cear, capear...
%cando o "a" ou "e" temático é tónico (ou casi).
\template{o3#.}
ar ado ada ados adas ando \newline
ares armos ardes aren \newline
arme arte arlle arnos arvos arlles \newline
alo alos ala alas \newline %incúe algunhas outras (coala)
ade ádeo ádea ádeos ádeas \newline
as amos ades an \newline
aba abas abamos abades aban \newline
aches astes aron \newline
es emos edes en \newline
ase ases ásemos ásedes asen \newline
\end{template}
\template{e3#.}
ar ado ada ados adas ando \newline
ares armos ardes aren \newline
arme arte arlle arnos arvos arlles \newline
alo alos ala alas \newline %incúe algunhas outras (coala)
ade ádeo ádea ádeos ádeas \newline
as amos ades an \newline
aba abas abamos abades aban \newline
aches astes aron \newline
es emos edes en \newline
ase ases ásemos ásedes asen \newline
\end{template}

\template{2#.}
%subir
i3mos i3des \newline
ía í3as í3an \newline	%2iamos, 2iades estropearía moitos verbos (ca-ia-mos... )
ín i3ches iu istes i3ron \newline
% O pres. subx. coincide con pres. ind. da 1ª
i3se i3ses í3semos í3sedes i3sen
%contraer, decaer, extraer
%Pres. subx. xa se divide ben
\end{template}
\newline
í3do í3da í3dos í3das

%Conxugación completa do verbo subir
\template{.su3b#.}
ir indo ido ida idos idas \newline
ires irmos irdes iren \newline
o es e imos ides en \newline
ía ías 2i3amos 2i3ades ían \newline %por eso incluímolos só aquí.
ín iches iu istes iron \newline
irei irás irá iremos iredes irán \newline
iría irías iriamos iriades irían \newline
a as amos ades an \newline
ise ises ísemos ísedes isen \newline
ide ídeo ídea ídeas
\end{template}

%para verbos rematados en -aer, permitimos a división
%cando a vocal temática tónica e aberta e, salvo inf.,
%non é a última sílaba.
\template{a3#.}
er endo \newline
eres ermos erdes eren \newline
erme erte erlle ernos ervos erlles \newline
elo elos ela elas \newline
ede édeo édea édeos édeas \newline
emos edes eron \newline
ese eses esemos esedes esen
\end{template}

%par-ti-a-mos, di-ci-amos, fa-ci-amos, etc., pero aso-cia-mos, etc.

%The new programs allow to input patterns at run time, so including
%the extras in the format is not right.
%\mkinput{glhyextr.tex}
\end{pseudopatterns}
\end{}
