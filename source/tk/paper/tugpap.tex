%**start of header
\catcode`@=11 % borrow the private macros of PLAIN (with care)

\newskip\ttglue

\font\elevenrm=cmr10 scaled \magstephalf % roman text
\font\eleventt=cmtt10 scaled \magstephalf % typewriter
\font\elevenit=cmti10 scaled\magstephalf
\font\ninebf=cmbx9
\font\ninerm=cmr9
\font\ninei=cmmi9
\font\ninesy=cmsy9
\font\ninett=cmtt9
\font\ninesl=cmsl9
\font\nineit=cmti9
\font\sixi=cmmi6
\font\sixsy=cmsy6
\font\sixbf=cmbx6
\skewchar\ninei='177 
\skewchar\ninesy='60 
\hyphenchar\ninett=-1 

\def\ninepoint{\def\rm{\fam0\ninerm}% set in nine point families
 \textfont0=\ninerm \scriptfont0=\sixrm \scriptscriptfont0=\fiverm
 \textfont1=\ninei \scriptfont1=\sixi \scriptscriptfont1=\fivei
 \textfont2=\ninesy \scriptfont2=\sixsy \scriptscriptfont2=\fivesy
 \textfont3=\tenex \scriptfont3=\tenex \scriptscriptfont3=\tenex
 \textfont\itfam=\nineit \def\it{\fam\itfam\nineit}% \it is family 4
 \textfont\slfam=\ninesl \def\sl{\fam\slfam\ninesl}% \sl is family 5
 \textfont\bffam=\ninebf \scriptfont\bffam=\sixbf
  \scriptscriptfont\bffam=\fivebf \def\bf{\fam\bffam\ninebf}% \bf is family 6
 \textfont\ttfam=\ninett \def\tt{\fam\ttfam\ninett}% \tt is family 7
 \tt \ttglue=.5em plus.25em minus.15em
 \normalbaselineskip=11pt
 \setbox\strutbox=\hbox{\vrule height8pt depth3pt width0pt}% strut for 11pt
 \let\sc=\sevenrm \normalbaselines\rm}

\font\scaps=cmcsc10 % for \LaTeX
\font\lb=logobf10
\chardef\bs=`\\ % backslash in a string
\def\;#1{{\it #1\/}} % The simplest and most useful of all.

\input mftmac
\parindent=1.5pc % restore after mftmac clobbered it

\font\ninetex=cmtex9 \hyphenchar\ninetex=-1
\def\finstring"#1"{\ninetex"#1"\egroup}

\def\Beginmft{\par\begingroup\ninepoint
% \chardef:=`\: \chardef!=`\! % \chardef==`\=
 \catcode`\:=12 \catcode`\!=12 \catcode`\==12
 \def\mathsemicolon{\mskip\thickmuskip}
 \let\;=\mathsemicolon
 % nine-point type:
 \def\bf{\fam\bffam\ninebf
  \def\_{\kern.04em\vbox{\hrule width.3em height .6pt}\kern.08em}%
  \ninebf}
 \textfont\bffam=\ninebf \scriptfont\bffam=\sixbf
  \scriptscriptfont\bffam=\fivebf
 \baselineskip=11pt
 \def\MF{{\manual hijk}\-{\manual lmnj}}
 \let\big=\ninebig
 \setbox\strutbox=\hbox{\vrule height8pt depth3pt width0pt}
 \rm
 \setbox\shorthyf=\hbox{-\kern-.05em}
 \parindent=0pt
 }
\def\Endmft{\par
     \endgroup} 

\input tkccode
\font\tentk=tkr10
% \font\elvntk=tkr11
\font\tenti=tkti10
% \font\elvnti=tkti11

\def\BeginTurkish{\par \begingroup
%   \chardef:=`\: \chardef!=`\! % \chardef==`\=
   \def\`{\char'43}\def\'{\char'100}%
   \catcode`\:=12 \catcode`\!=12 \catcode`\==12
   \sfcode`\:=1000 \sfcode`\!=1000 \sfcode`\==1000 % Frenchspacing after
%                                                    these
   }
\def\EndTurkish{\par %let's be quite certain about this!
   \endgroup}
\def\verbatim#1{\begingroup\ninepoint \frenchspacing
  \def\do##1{\catcode`##1=12 } \dospecials
  \catcode`\:=12  \catcode`\!=12  \catcode`\==12
  \parskip 0pt \parindent 0pt
  \catcode`\ =\active \catcode`\^^M=\active
  \tt \def\par{\ \endgraf}\overfullrule=0pt \obeylines \obeyspaces
  \input #1 \endgroup}
% a blank line will be typeset at the end of the file;
% if you're unlucky it will appear on a page by itself!
{\obeyspaces\global\let =\ }

\font\titlefont=cmr17
\font\namefont=cmcsc10 scaled \magstep1
\font\twelvebd=cmbx12
\font\sixrm=cmr6

\vsize=43pc
\hsize=29pc
\parindent=16pt
\nopagenumbers
\normalbaselines
\footline={\hss\sixrm\folio\hss}
\def\makefootline{\baselineskip3pc\line{\the\footline}}
\def\plainoutput{\shipout\vbox{\makeheadline\pagebody\makefootline}%
  \advancepageno
  \vsize=45pc
  \ifnum\outputpenalty>-\@MM \else\dosupereject\fi}
\def\makeheadline{\vbox to\z@{\vskip-22.5\p@
  \line{\vbox to8.5\p@{}\ifnum\count0=1
         \else\ifodd\count0
                \hfill\the\doctitle\ignorespaces
             \else \the\authorname\hfill\fi\fi}\vss}\nointerlineskip}

\newtoks\doctitle  \newtoks\authorname

\doctitle={The Ottoman Texts Project}
\authorname={Walter Andrews and Pierre MacKay}
% Added to the general format file
% an addition needed by mftmac
\def\ninebig#1{{\hbox{$\textfont0=\tenrm\textfont2=\tensy
   \left#1\vbox to7.25pt{}\right.\n@space$}}}
\catcode`@=12 % at signs are no longer letters
\tenrm
\begingroup
\titlefont
\the\doctitle\hfil
\vskip 3pc
\namefont\the\authorname\hfil
\tenrm
\vskip 1pc
{\obeylines
Department of Near Eastern Languages and Civilization
University of Washington
Seattle, Washington 98195
}
\vskip 3pc
\line{\hfill\twelvebd ABSTRACT\hfill}
\vskip 3pc
\leftskip 4pc \rightskip 4pc
The Turkish orthographic reform of 1928, which required the abandonment
of Arabic script in favor of a Latin letter alphabet, was accompanied
by a cultural rejection of all literature from the
Ottoman period of Turkish history.  As a result, only a small part of 
Ottoman Turkish literature has been made available in scholarly editions
in the new orthography.  The Ottoman Texts Project is a cooperative
effort of Turkish and North American scholars to provide new editions
of these works using popular low-priced personal computer systems and
standard general purpose software.  This paper describes an approach
based on the adoption of \TeX\ as the preferred output system for
publication.
\par
\endgroup
\vskip 3pc
%**end of header
\font\eightrm=cmr8
\def\sc{\eightrm}

\font\huge=cmr17 scaled \magstep2
\def\dropinitial#1#2{\def\biginitial#1{{\huge#1}}%
  \def\makeinitial#1{\setbox8\hbox{\strut\vbox to 1.3ex
    {\hbox{\biginitial#1}\vskip -4pc plus 3.5pc minus 3.5pc}}}%
  \makeinitial#1%
  \ifdim\parindent>1.3\wd8\dimen8=\parindent
     \else\dimen8=1.3\wd8\fi
  \hangindent=\dimen8\hangafter=-2
  \noindent
  \strut\hskip-1\dimen8\box8{\sc#2}}%

\noindent
The Ottoman Texts editing and typesetting project
represents an attempt to provide a simple, low-cost
system for the entry, editing, and typesetting of
transcribed [romanized] Ottoman Turkish texts.  The
purpose of developing such a system was to take advantage
of the increasing availability of microcomputers
world-wide and to induce the editors of Ottoman texts--
especially Turkish editors--to employ electronic media
for their editing tasks.  The benefits to scholars of
having a large corpus of texts available in
machine-readable form seem obvious, but overcoming
``technology cringe'' on the part of scholars whose
devotion to medieval literature stems in large part from
a strong conservative-traditionalist ideological bent is
no small task.  Nonetheless, the rewards of converting a
significant number of such scholars would be quite high.
The vast majority of significant Ottoman Turkish texts
await up-to-date editing and the suggested technological
change could have a major impact on the speed and
accuracy of the editing process as well as on the
development of lexicographical tools and on many areas of
literary and linguistic study.

        The situation in Ottoman studies that makes a
switch to electronic media especially attractive at this
time is rather complex and demands some historical
introduction.  From its earliest years at about the
beginning of the 14th century until early in the 20th
century, the Ottoman dialect of Turkish was written in
the Arabic script.  The political decline of the Ottoman
Empire from its pinnacle of world power in the 16th
century to its status as a moribund, defeated ally of
Germany following World War I, was arrested in the first
three decades of this century by a political and
ideological revolution that saw the establishment of a
Turkish Republic and an accompanying rejection of the
literary, cultural, and religious institutions of the
Ottoman past.  One aspect of the cultural revolution was
the adoption of a latin letter alphabet for Turkish, a
change which had among its
consequences the expansion of literacy beyond a small
elite circle to the general populace, a conscious effort
to simplify the written language, and a resultant major
decline in the ability to read and comprehend the Ottoman
literary language in any of its forms.  The ethos of the
early years of the Republic, to which the Ottoman Empire
appeared as decadent and its culture as derivative, also
meant that, at a time when the scholarly edition of older
texts was becoming a growing concern in other parts of
the world, in Turkey interest in things Ottoman,
including Ottoman texts, was considered backward,
anti-nationalist, counter-revolutionary, and
wrong-headed.  As a result, very few texts were
adequately edited and the population in general was
further cut off from its historical past.  Since the
Second World War, however, there has been an increased
scholarly interest in Ottoman texts and in the
transcription and edition of such texts.  This interest
has grown with the growth of a tolerance for some
reemergences of older ethical, religious, and cultural
practices and attitudes.

        It is clear that the particular situation in
Turkey today lends itself to the adoption of editing
methodologies that take advantage of computer technology:
there is a large cadre of well-educated persons with very
positive attitudes toward technological innovation; the
Latin alphabet is used [with modifications for Ottoman
transcription]; most of the basic editing work remains to
be done; there is already great interest in the types of
concordancing, indexing, lexicographical analysis, etc.
that can be most easily done by computers.  Nonetheless,
Ottoman studies is still an area that attracts persons
who would be least likely to welcome technological
innovation and so any change would need to bring
immediate and obvious benefits.  When the editing project
was initiated, it was decided that the result should have
the following characteristics:

\medskip
\noindent1.\quad It should be easy to use even for the most
unsophisticated user.

\noindent2.\quad It should be adaptable to many different
circumstances and should be easily supportable.

\noindent3.\quad It should obviously eliminate the need for
more than one entry of the basic text.  [This is, of
course, common to all computer word-processing systems
but it is such a major departure from the usual round of
draft typings that its benefits must be emphasized to
those who have not experienced it.]

\noindent4.\quad  It should be capable of producing typeset
camera-ready copy for printing. [This is a major
potential benefit even in Turkey where the costs of more
labor-intensive typesetting methods are growing rapidly.]

\medskip

        The project developed in several stages and was
not without its problems and false starts.  The first
stage involved convincing a noted Turkish scholar and
respected editor of Ottoman texts to come to the
University of Washington to attempt to edit the collected
poems of a 16th century Ottoman poet using the IBM XT
already employed by the Department of Near Eastern
Languages and Civilizations for the development of
Turkish character-sets.  Scholarly processes being what
they are it turned out to be easier to bring the scholar
than to have the necessary word-processing capabilities
ready when he arrived.  As a result, a rather cumbersome
combination of Microsoft's WORD, Rosesoft's ``smart key''
program [PROKEY], and a series of BASIC programs
developed by Robert Blum of the UW administration was
used to enable the Turkish visitor to input and edit
about 90\% of the poems in the collection [over 500 poems]
in about three months.  The editor, who had had no
previous experience of computers and no particular liking
or aptitude for them, was an eager and willing convert to
the process.  Prior to his departure, we were also able
to employ a simple translation program which converted
the character-set designed for the XT to \TeX\ notation
and, subsequently, to produce a typeset sample of the
edited text on the SUN
minicomputer.  The reaction of our visitor to the
results of this process, which was carried out with the
help of two fellow scholars without the intervention of
typists or typesetters, was pure delight and amazement.

        In the ensuing months the project has been
considerably refined and improved.  With the invaluable
assistance of the UW Humanities and Arts Computing
Center and its resident character-sets guru, Gerald
Barnett, we have been able to develop a word-processing
system that is simple, efficient, flexible, and low-cost.
The system is based on Quicksoft's PC WRITE program used
with EGA/VGA and compatable graphics hardware.\footnote{$^*$}{NOTE: At
present, the system produces a host of irritating ``ghost
diacritics'' when used with the IBM PS 2 graphics--these
are a distraction more than a real hinderance but, as
yet, we have no idea why they occur.}  The advantages of
PC WRITE for this kind of word-processing are numerous
but it is worth mentioning a few in some detail.

        Given the goal of making this technology widely
available among scholars and students [especially among
foreign students and scholars], the fact that PC WRITE is
low-cost, share-ware [\$89.00 with full support] makes it an
attractive alternative.  Moreover, PC WRITE permits
virtually limitless customization of keyboards, fonts,
printer controls, etc. in a manner accessible to persons
without any knowledge of programming or programming
languages.  Using a simple set of programs--a program
designed at Duke University for the creation of
characters for display on an EGA driven monitor and a
program being developed by Gerald Barnett of the UW for
the production of downloadable printer fonts--we have
been able to produce a word processing system that can
display and edit an extended IBM character-set, which
will allow the use of modern [roman alphabet] Turkish,
the romanized transcription of Ottoman Turkish [Arabic
alphabet], and a full English characters font.  One can
also switch instantly between a standard IBM
keyboard, an IBM keyboard adapted to Turkish
characters, and the standard Turkish keyboard with
extensions for the Ottoman character-set.  In addition,
the system supports draft printing on the IBM Pro Printer
and letter quality printing on the NEC and Toshiba 24 pin
printers [with the use of a bi-directional tractor].

        The extended IBM character-set uses 8 of the
special European characters, 32 special Ottoman
transcription characters [on ASCII codes 192--223], and 9
special modern Turkish characters [on ASCII codes
225--233], as well as the full English set.  All of the
modern Turkish characters appear as characters on the
modified standard keyboard.  The Ottoman Turkish
characters [standard English characters with diacritics]
are called up by two-key sequences.  For example, a ``d''
with a dot under it is produced by striking ``/'' followed
by ``d''; all other special Ottoman characters are produced
by the same sequence [``/''$+$``character''].  Keyboard
arrangement and the particular character used to call up
the special characters can be easily modified to suit the
preferences of the user.

        One fortunate aspect of the print control features of PC
WRITE for this project
is that the print control program can be set up to
support two different fonts for each character.
Therefore, \TeX\ notation can be provided as an alternative
for each character and translation from the usual
word-processor font to \TeX\ notation can be done
automatically by simply printing in the \TeX\ input character set to
another file.  Because PC WRITE produces ``clean'' ASCII
files, the material is immediately ready for typesetting
in whatever \TeX\ environment is being used.

\medskip
\centerline{{\elevenrm Accented character sets in \TeX}}
\smallskip

\noindent
In the few years since 
the official release of \TeX, a number of
attempts have been made to adapt the program to languages other than
English.  The best known successes have depended on adaptations of the
program itself, partly because the standard release of \TeX\ can
support only one system of hyphenation at a time, which makes a truly
bilingual document quite difficult to produce.  These adaptations may
be broadly classed as program-based extensions of the language.  The
extension which is most obviously necessary is the addition of a
primitive which can control the switch between one predigested
hyphenation pattern and another.  Michael Ferguson's bilingual CNRS-\TeX,
which was initially developed for an environment in the province of
Quebec, where French and English are constantly intermingled, is one
of the outstanding developments in this class of adaptations, and
there are others as well.

A second extension is needed to get around the problem of hyphenation
in languages which make use of diacriticals and accents.  The basic
form of \TeX\ will reject any word containing an accent from the
evaluation routine
which normally looks for acceptable hyphenation breaks.  In effect,
any word with an accent is treated as if it were an unbreakable
horizontal box, and is not evaluated for hyphenation at all.  This can
make line-breaking very difficult, and several users of \TeX\ have
found it necessary to introduce a loop into the program so that
accents and diacriticals will be stripped out just before the entry to
the hyphenation routine, and then returned to their remembered positions
after the discretionary hyphen nodes have been inserted into the word.

The disadvantage of both these systems is that the adapted program is no
longer \TeX.  It is often possible to add the extra features in such a way 
that the resultant program will produce {\tt DVI} files that are
indistinguishable from those generated by \TeX, but the extra features
are not generally available on all systems which run \TeX, and the
user is often excluded, therefore, from some of the most popular small
system versions of \TeX.

An alternative solution to the problem of accented languages, though
not of bilingual hyphenation patterns, is a font-based, rather than a
program-based approach.  Font characters may be generated with the
accents already applied, and mapped into unused or little-used areas
of the normal Computer Modern font table.  If these characters are
then supplied with an appropriate \TeX\ {\tt\bs lccode} value, the
hyphenation loop will recognize them as part of a sequence capable of
being hyphenated.  For a monolingual application in a language which
makes intensive use of accents and diacriticals, this can be an
attractive approach, especially when there are reasons for wishing to
preserve the ability to make use of small system versions of \TeX.  
This is the approach we have taken for Turkish \TeX.

Turkish provides a delightfully vivid set of examples of accentuation
and hyphenation.  The Latin-letter character set which has been in
use since the orthographic reform of 1928 is extended, even in Modern
Turkish, by means of a considerable number of diacriticals and accents.  A
diligent search through the modern dictionary will produce several
five- and six-letter words in which every character is accented, and an
intensive search might come up with words as much as nine letters long
with every character accented.  In critical editions of Ottoman texts,
the number of accents more than doubles.  Modern Turkish knows only
the accented and unaccented pair of letters `{\bf s}' and `{\bf\c s}', but
Ottoman Turkish has `{\bf s}', `{\bf\c s}', `{\bf\d s}' and `{\bf\b s}', which
represent four completely distinct characters in the Arabic alphabet.
The letter `{\bf h}' shows almost as much variety, and so do several
others.  Our Ottoman Turkish font has twenty-seven accent and letter
composites, in addition to the basic twenty-six simple Latin letters.
Moreover, all composites can exist in upper case forms as well as in
lower case.

When a character set is as heavily accented as this, it is desirable
to make sure that the accents are positioned over their letters as
exactly as possible.  The {\tt\bs accent} primitive in \TeX\ does a
remarkably good job of positioning accents, but it depends on a very
general algorithm, and tends to place accents exactly centered over
or under the affected character, no matter what the appearance of
that character may be.  Donald Knuth recognized this limitation in
the very earliest stages of the development of \TeX, and has
consistently recommended that frequently used combinations of
character and accent be developed as composite single images in the
font.  The center of a character is not always the best visual
position for an accent; top accents should often be slipped just a
bit to the right, and bottom accents just a bit to the left of the
mechanically defined centerline of the character.  Height and depth
of accents are similarly subject to aesthetic judgement.  
The{\tt\bs accent} primitive of \TeX\ works very well indeed for
sparsely occurring accentuation, but not so well when accents occur
in every second word.

The problem of hyphenation in Turkish is even more striking.  Turkish
is known as an ``agglutinating'' language, which means, in effect that
each discrete logico-syntactic qualification of a basic word is
expressed in a single syllable tacked onto all the other syllables in
the word.  At the same time, it is a language in which consonant
clusters are virtually unknown.  A Turkish word is made up of simple
open and closed syllables, of the form {\tt cv} or {\tt cvc}, and in
native words there is not even the distinction between long and short
vowels.  The result is a language in which word-length tends to be
greater than it is even in English, and where, as a result,
hyphenation is often necessary.  The hyphenation rules are inherited
from the syllabification of Arabic.  A syllable is assumed always to
consist of an initial consonant (even when that consonant is no longer
written) and to terminate in a vowel or in the next unvowelled
consonant.  This pattern is followed so absolutely that it is
permitted to break up native Turkish suffixes.  The plural suffix \;{
-ler-} will be hyphenated as \;{-le-rine} in an environment where
the {\tt -cv-cv-cv} pattern predominates.

A set of hyphenation patterns for Turkish will therefore be quite
simple to produce, but it will have no effect on most Turkish words
unless something is done about the problem of accents.  A word such as
\;{\c cektirilebilecek} ought to provide six discretionary
hyphenation nodes: \;{\c cek-ti-ri-le-bi-le-cek}, but the {\tt\bs
accent} primitive applied to the first letter will guarantee that the
standard version of \TeX\ gives up any attempt to hyphenate it at 
all.\footnote{$^*$}{The word is a future participle, and describes 
something as being
capable of being extracted at some time in the future--like a tooth.
A morphological division of the word would produce a very different
hyphenation pattern, \;{\c cek-tir-il-e-bil-ecek}, with only five nodes.}
If the initial letter `{\bf \c c}' were a single character in a special
font, and were provided with an {\tt\bs lccode} value, the {\tt\bs
accent} primitive would no longer appear, and the word could be
evaluated for hyphenation.

Since the majority of \TeX\ users will never have to deal with {\tt\bs
lccode}s at all, a word of explanation is in order here.  \TeX\
is designed to take care of the problems of typesetting in a
general manner, independent of the language of the text to be
set.  The program recognizes that while many languages have
paired upper and lower case character sets, not all do, and the
order of the basic text character set may not be that of the
Latin alphabet.  For this reason, specific upper and lower case
pairings are not built into the program, but are supplied by
macro definitions in {\tt plain.tex}.  Like all other
definitions in plain.tex, they may be replaced, and it is quite
possible to dispense with plain.tex altogether, and substitute
another basic format file such as {\tt sadece.tex}, {\tt
franc.tex}, {\tt einfach.tex} or {\tt sketo.tex}.  (Knuth
insists, for obvious reasons, that the one thing you may not call it is
``plain.tex.'')  If additional characters such as the accented letters
of Turkish are made part of the basic input coding table, then they
are likely to exist in upper and lower case pairs.  Each lower case
code is given itself as a lower case {\tt\bs lccode}, and the code of
its upper case equivalent as its {\tt\bs uccode}.  These can be used
to force conversion from one case to the other, but the {\tt\bs
lccode} serves an additional purpose.  When \TeX\ enters the program
loop which searches for discretionary hyphen nodes in each word, it
first unpicks all ligatures such as {\bf ffi} and then evaluates the
resultant list from the beginning,
working on any given word only so long as every character it
finds has a valid {\tt\bs lccode}.  Any node that is not a simple
character with a valid {\tt\bs lccode} causes the routine to terminate;
the sequence so marked is supplied with no discretionary
hyphen nodes at all, and therefore cannot be broken by the
line-breaking algorithm.  This is what prevents hyphenation in the
case of the Turkish word given above.

\medskip
\centerline{\elevenrm Input Code Interpretation}
\smallskip

\noindent
The Turkish text-editing system described above
is driven from a keyboard mapped to conform as
closely as possible to the standard Turkish typewriter keyboard.
This mapping is not used directly in the design of the Ottoman
Turkish font and, in its present form,
is isolated from the actual \TeX\ input.  After the
raw input has been corrected, it is passed through a filter which
converts the accented characters into character pairs (or, in a very
few instances, into \TeX\ command sequences).  These pairings are
based on a proposal made more than ten years ago at the Orientalist
Congress held in Paris, in 1974.  Owing to the extraordinary richness
of the Ottoman Turkish character set, it has been necessary to extend
the old proposal, but it still retains the original principles, which
are closely associated with the coding scheme used by the Onomasticon
Arabicum project.  The Onomasticon Arabicum uses a post-positive dot
and a post-positive hyphen to indicate diacriticals, which is
acceptable in a data-base of names, but not in continuous prose text.
To provide the indications for Ottoman Turkish diacriticals, 
we have taken over the exclamation
point `{\tt!}', the equals sign `{\tt=}', and the colon `{\tt:}'.

The exclamation point is used for all the ``emphatic'' letters of the
Arabic alphabet (the alphabet in which Turkish was written until
1928).  These are the letters \;{\d Dad} (usually pronounced as `{\bf z}' in
Turkish, and hence paired with a non-Arabic letter known as \;{\.Zad}), 
\;{\d Sad}, \;{\d Ha'}, \;{\d Ta'} and \;{\d Za'}.  
The equals sign is used for all the
consonants which are represented in Latin-letter transcriptions by a
letter with a bar under, such as {\bf\b d} (\;{dhal}), more 
commonly written in
Turkish as `{\bf\b z}', and also for vowels with a macron or, following the
Turkish convention, a `hat' accent, and similar forms, chosen like the
cupped `{\bf\u g}', because the equals sign is visually closer than the colon
is.  (Moreover, the colon is needed for a different variety of the
letter `{\bf g}'.)  The colon is a catch-all for everything else, but works
out rather well visually, as it happens.  The three post-positives are
not accents, but regular characters, which use the \TeX\ convention of
ligatures to invoke accented characters from the font, just as the
second `{\bf f$\,$}' in the normal \TeX\ `{\bf ff}' 
ligature pair does.  If a standard
Latin-letter character does not have an associated ligature table in
the font, a following colon will be unaffected.  
Thus, the letter `{\bf o}',
when followed by a colon will produce `{\bf\"o}', but the letter `{\bf e}' when
followed by a colon will produce `{\bf e:}'.  The equals sign is returned to
its normal function in math mode, and the colon and exclamation point
can be invoked by the command sequences {\tt\bs:} and {\tt\bs bang}
when the simple character will not work.

This set of conventions produces an input file which can, if
necessary, be edited on a ordinary terminal lacking the special
Turkish character features, and which a Turkish speaker can become
accustomed to without too much difficulty.  When coupled with a
well-designed macro file and a rewritten hyphenation table, it
provides the possibility of naturalizing a \TeX\ environment into
Turkish without any large investment in special purpose hardware and
rewritten versions of non-standard (non-)\TeX.

\medskip
\centerline{\elevenrm The Font}
\smallskip

\noindent
Donald Knuth's Computer Modern fonts come with a wide range of accents, which
cover most of the requirements for Turkish.  The only obvious lack is
the flat cup which is used under both upper and lower case `{\bf h}' as
an aesthetic variant for the simple bar under the letter.  All the
existing accents in Computer Modern are designed for consistency 
with the stroke-weights and proportions of the underlying alphabetic
characters, and it is therefore very desirable to
retain the details of this design in any associated font of accented
characters.  The vertical and horizontal positions
may be altered and, for other languages than Turkish, the angle of
acute and grave accents over upper case letters, but the basic
proportions of each accent or diacritical remain unchanged.  This is
achieved by taking over the entire text of the Computer Modern
character file {\tt accent.mf} and converting the {\tt beginchar
$\ldots$ endchar} pairs to {\tt def} and {\tt enddef}.  It is
not quite so easy as that, but the process is essentially mechanical,
and guarantees the preservation of all the essential design details for
each accent.  (The flat cup under `{\bf h}' is based on the slavic tie
accent, turned upside down.)  The resultant file, {\tt accdef.mf}, 
is now full of ``definitions'' which can be
invoked as part of the program file for composite characters.  
Positioning, however, can not be entirely taken care of in the 
{\tt accdef.mf} file.  The accents in {\tt accent.mf} 
are, for the most part, designed with a fixed reference
point at the top of the image, but correct positioning usually
requires a knowledge of where the bottom edge will be.  It is
herefore necessary to take some of the calculations from the accent
definitions, and incorporate them into the description of the
underlying character.  For example, the superscript dot accent in the
Computer Modern font is produced as follows.
\medskip
\Beginmft
$\2{iff}\\{ligs}>0\?\2{cmchar}\7"Dot accent";$\par
$\2{numeric}\\{dot\_diam}\SH ;\ \\{dot\_diam}\SH =\1{max}(\\{dot\_size}\SH ,%
\\{cap\_curve}\SH );$\par
$\2{beginchar}(\1{oct}\7"137",5u\SH ,\1{min}(\\{asc\_height}\SH ,\frac10/{7}%
\\{x\_height}\SH +.5\\{dot\_diam}\SH ),0);$\par
$\2{define\_whole\_blacker\_pixels}(\\{dot\_diam});$\par
$\2{italcorr}h\SH \ast \\{slant}+.5\\{dot\_diam}\SH -2u\SH ;$\par
$\2{adjust\_fit}(0,0);$\par
$\2{pickup}\\{tiny}.\\{nib};\ \\{pos}_{1}(\\{dot\_diam},0);\ \\{pos}_{2}(\\{dot%
\_diam},90);$\par
$x_{1}=x_{2}=.5w;\ \\{top}\,y_{2r}=h+1;$\par
$\2{if}\\{bot}\,y_{2l}<\\{x\_height}+o+\\{slab}\?y_{2l}:=\1{min}(y_{2r}-%
\\{eps},\\{x\_height}+o+\\{slab}+.5\\{tiny});\3{fi}$\par
$y_{1}=.5[y_{2l},y_{2r}];\ \\{dot}(1,2);\ \9 dot\par
$\2{penlabels}(1,2);\3{endchar};$\par
\Endmft
\medskip

\noindent
The corresponding {\tt accdef.mf} definition is

\medskip
\Beginmft
$\2{def}\\{dot\_accent}(\2{suffix}\$,@)(\2{expr}\\{dotY\_shift})=$\par
$\2{save}@;$\par
$\2{forsuffixes}\\{\$\$}=@,@_{\_}\?\2{transform}\\{\$\$};\3{endfor}$\par
$\2{numeric}\\{dh}\SH ;\ \\{dh}\SH :=\1{min}(\\{asc\_height}\SH ,\frac10/{7}%
\\{x\_height}\SH +.5\\{dot\_diam}\SH );$\par
$\2{define\_whole\_blacker\_pixels}(\\{dh},\\{dot\_diam});$\par
$\2{pickup}\\{tiny}.\\{nib};\ \\{pos}_{@\_1}(\\{dot\_diam},0);\ \\{pos}_{@\_2}(%
\\{dot\_diam},90);$\par
$x_{@\_1}=x_{@\_2}=x_{\$};\ \\{top}\,y_{@\_2r}=\\{dh}+1;$\par
$\2{if}\\{bot}\,y_{@\_2l}<\\{x\_height}+o+\\{slab}\?y_{@\_2l}:=\1{min}(y_{@%
\_2r}-\\{eps},\\{x\_height}+o+\\{slab}+.5\\{tiny});\3{fi}$\par
$y_{@\_1}=.5[y_{@\_2l},y_{@\_2r}];$\par
$\2{numeric}\\{dot\_span};\ \\{dot\_span}=\\{dh}-\\{bot}\,y_{@\_2l};$\par
$@=\\{identity}\2{if}\\{dotY\_shift}<>0\?\6{shifted}(0,\\{dotY\_shift}+\\{dot%
\_span})\3{fi};$\par
$\2{for}n=1,2\?\2{forsuffixes}e=l,,r\?$\par
\quad$z_{@}[n]e=z_{@\_}[n]e\6{transformed}@;\3{endfor}\3{endfor}$\par
$\\{dot}(@_{1},@_{2});\ \9 dot\par
$\2{penlabels}(@_{1},@_{2});\3{enddef};$\par
\Endmft
\medskip

To get this into position over the letter `{\bf o}', requires the
following program text,

\medskip
\Beginmft
$\2{cmchar}\7"The letter dotted o";$\par
$\\{dot\_sharp\_values};$\par
$\2{beginchar}(\1{oct}\7"025",9u\SH ,\\{dot\_top}\SH ,0);$\par
$\2{italcorr}\frac1/{3}[\\{x\_height}\SH ,\\{asc\_height}\SH ]\ast %
\\{slant}-.5u\SH \2{if}\\{serifs}\?+.25\\{dot\_diam}\SH \3{fi};$\par
$\2{adjust\_fit}(\2{if}\\{monospace}\?.5u\SH ,.5u\SH \2{else}\?0,0\3{fi});$\par
$\\{penpos}_{1}(\\{vair},90);\ \\{penpos}_{3}(\\{vair}',-90);$\par
$\\{penpos}_{2}(\\{curve},180);\ \\{penpos}_{4}(\\{curve},0);$\par
$x_{2r}=\1{hround}\1{max}(.5u,1.25u-.5\\{curve});$\par
$x_{4r}=w-x_{2r};\ x_{1}=x_{3}=.5w;\ y_{1r}=\\{x\_height}+\1{vround}1.5\\{oo};\
y_{3r}=-\\{oo};$\par
$y_{2}=y_{4}=.5\\{x\_height}-\\{vair\_corr};\ y_{2l}:=y_{4l}:=.52\\{x%
\_height};$\par
$\2{penstroke}\\{pulled\_arc}_{e}(1,2)\AM \\{pulled\_arc}_{e}(2,3)$\par
\quad${}\AM \\{pulled\_arc}_{e}(3,4)\AM \\{pulled\_arc}_{e}(4,1)\AM \1{cycle};\
\9 bowl\par
$\2{numeric}\\{dot\_shift},\\{dot\_top};$\par
$\2{define\_whole\_blacker\_pixels}(\\{dot\_diam},\\{dot\_top});$\par
$\\{dot\_shift}=0;\ \9 in this case, the position happens to be correct\par
$x_{7}=x_{1}-.8\\{dot\_diam};\ x_{8}=x_{7}+1.6\\{dot\_diam};$\par
$\\{dot\_accent}(7,a,\\{dot\_shift});$\par
$\\{dot\_accent}(8,b,\\{dot\_shift});$\par
$\2{penlabels}(1,2,3,4,7,8);\3{endchar};$\par
\Endmft
\medskip

\noindent
in which the line

\medskip
\Beginmft
$\\{dot\_sharp\_values};$\par
\Endmft
\medskip

\noindent
expands to a macro

\medskip
\Beginmft
$\2{def}\\{dot\_sharp\_values}=$\par
$\2{numeric}\\{dot\_diam}\SH ;\ \\{dot\_diam}\SH =\1{max}(\\{dot\_size}\SH ,%
\\{cap\_curve}\SH );$\par
$\2{numeric}\\{dot\_top}\SH ;\ \\{dot\_top}\SH =\1{min}(\\{asc\_height}\SH ,%
\frac10/{7}\\{x\_height}\SH +.5\\{dot\_diam}\SH );$\par
$\!\3{enddef};$\par
\Endmft
\medskip


\noindent
which repeats some of the calculations made in the definition of the dot
accent.


The composites that result from this programming effort look, for the
most part, identical to the results of the application of the {\tt\bs
accent} primitive to characters in the regular Computer Modern fonts.
The one major difference comes in the shape of the ``hat'' accent over
the letter `{\bf i}'.  In this instance, the accent would spread beyond the
left and right side bearings of the underlying character and mess up the
letter spacing if it were
not pinched in, so a special narrow hat accent is provided for `{\bf i}'.  
The proportions of each stroke remain essentially the same as
those in the original model, but they form an acute angle over the top
of the letter.  Except in the case of this character and some of the 
uniquely Turkish dotted uppercase letters, 
it will probably be difficult to distinguish the two styles of accent 
in the final printed version even when they are
intermingled in the same text.

The creation of the composite characters is only the first stage in
the development of the font.  Next, the italic correction must be set
for all the italic and slant fonts.  This is the spacing that may be
added to the right side of any slanted character to prevent it from
running into something like a non-slanted closing parenthesis.  There
does not seem to be any way except visual inspection to discover an
acceptable italic correction.  One wants a fairly simple, general
calculation, but one which will do rough justice
to all slanted versions of the character.  There were more proof
copies generated to get the italic correction right than for any other
feature of the font.  (In the absence of any accessible system on which
proofs could be displayed on the screen, a great many paper proofs had
to be generated.)

Following this comes the generation of ligature and kerning tables,
which are necessarily quite large, and need to be carefully worked out
since there is only a finite region of a {\tt tfm} file that can be
devoted to them.  The smaller of the two ligature tables, for the
italic fonts, is shown in appendix A.  It still needs one further refinement;
the kernings appear in the order of English letter frequency, and it
might be possible to gain a little efficiency by rearranging some of
them.  Notice that the `{\bf f$\,$}' ligatures are altogether eliminated.
In Turkish it is essential to retain the distinction between the
dotted and the undotted `{\bf i}, which cannot be done if the `{\bf
fi}' ligature is used.  The problem that arises, in fact is to provide
adequate separation between the dotted `{\bf i}' and a preceding `{\bf
f$\,$}'.

In addition to the accented characters, it was necessary to design
three additional characters for Ottoman Turkish.  The simplest is
a dot at about the bar height of lower case `{\bf e}'.  This is
used for a type of Persian suffix known as ``izafet,'' which is
very common in Ottoman texts.  The remaining two characters are
representations of the Arabic letters ``Ayn'' and ``Hamza,'' which
are conventionally represented by opening and closing single
quotes in most fonts.  The ``lazy man's \;{`ayn}'' (as just
illustrated) is acceptable for the occasional reference, but not
for extensive literary texts.  Ayn is not an accent, it is a
regular consonant of the Arabic alphabet and Hamza, though it can
be omitted in many positions is also a consonant.  What is needed
is a pair of characters which are clearly distinguishable from
single quotes, but sufficiently like them to conform with the
general appearance of Computer Modern.  The programs shown below,
draw on the same standard definition as is used to generate the
single quotes, but alter the position and the proportions.  The
bulb is uppermost in both instances, and is somewhat smaller than
the bulb of the close quote.  The tail is brought out further from
the side of the bulb, and is tucked more tightly under.  The
versions for slanted and italic fonts use some special
transformations to insure that the {\tenti \`ayn} (that was the
character from the Ottoman font) is correctly formed.  In
effect, the character is built out to the left of the centerline,
with a reverse slant, and then reflected back into the normal letter space.
The program for these characters is Appendix B.

A complete passage from our first proposed critical text edition is
given below, first in \TeX\ input coding, and then as typeset.  The
text from which this passage was extracted runs to twelve pages, and
was set without the benefit of a properly rewritten hyphenation table.
By good luck, most of the English pattern hyphenations turned out to
correspond with acceptable Turkish hyphenations, but it will certainly
be necessary to make up a proper Turkish hyphenation table in the near
future.  When that is done, and an appropriate set of formatting
macros has been written to isolate Turkish text from non-Turkish text
and math mode, we will have a Turkish language adaptation of \TeX\
which can be exported onto any small \TeX\ system, with no alteration
of the program whatsoever.  The full range of standard Computer Modern
font styles will be available, and will blend in perfectly with the
normal unaccented library of Computer Modern fonts.  We will not have
a truly bilingual version of \TeX, but for a predominantly Turkish
language environment we will be offering a cheaper and more accessible
monolingual font-based adaptation.

\bigskip
{\verbatim {Tktexinput} }

\bigskip
\BeginTurkish
\tentk
\`A=s:ik!lik! zama=ninda \`is:k! va=sit!asi ve s:eyda=lik! \`a=lemi:nde
s:evk! vesi=lesi:, vus!lat eyya=minda mah!abbet muk!tez:a=si, fi:ra=k!
gu:nleri:nde h!urk!at i:k!ti:z:a=si, baha=r mev\-si:mi:nde s!oh!bet
germi:yyeti:, mah!bu=blar mecli:si:nde s:ara=b keyfi:yyeti:, ca=na=neler
i:bra=mi ve \`a=s:ik!lar i:k!da=mi ve fuz:ala= mus!a=h!abeti: ve \`uk!ala=
i:lti:fa=ti, ehl..i: di:ller rag:beti: ve t!a=li:bler mi:nneti: i:le di:du:gu:
ebya=t ve es:\`a=r, ki: her bi:ri:nu:n= lat!i=f ma\`a=ni=si: ca=m..i naz!ma
s:ara=b..i rengi=n ve s:i=ri=n h=aya=la=ti bezm..i: s!afa=da nuk!l..i:
s:ekkeri=n olup mu\`a=s:i:ra=n..i mecli:s..i: z=evk! bu meyh=a=nenu:n=
ba=deci:si: ve h!ari=fa=n..i bezm..i: s:evk! bu ka=s:a=nenu:n= sebu=-kes:i:
olmis:lardi.  K!alem..i: i:\`ti:z=a=r bu h=a=me..i: i:nki:sa=r i:le bu evra=k!a
tah!ri=re i:k!da=m ve bu ecza=ya tast!i=re i:hti:ma=m go:sterdi:.  
\EndTurkish
\tenrm
\vfil\eject

\strut\vskip .75in
\centerline{{\elevenrm Appendix A}}
\smallskip
\centerline{{\elevenrm The {\eleventt turkit.mf} driver file}}
\bigskip
\Beginmft
\input turkit.tex
\Endmft

\vfil\eject

\strut\vskip .75in
\centerline{{\elevenrm Appendix B}}
\smallskip
\centerline{{\elevenrm The {\tt aynhmz.mf} file}}
\Beginmft
\input aynhmz.tex
\vfil
\Endmft








\bye
