\usemodule
	[s-lan-04]

\definecolor
	[hyphenation:mn]
	[b=.9]
\definecolor
	[hyphenation:mn-lmc]
	[r=.9]

\installlanguage
	[mn]
	[patterns=mn]
\installlanguage
	[mn-lmc]
	[patterns=mn-lmc]
\mainlanguage
	[mn]

\setupbodyfont
	[gentium]

\starttext

\startcomparepatterns[mn,mn-lmc]
Монгол хулан (Equus hemionus hemionus) хулангийн дэд зүйл. Говийн хулан буюу Зиггетай (Equus hemionus luteus) хэмээх нэртэй ижил утга илтгэнэ. Монгол болон хойд Хятадад тархсан, ан авын улмаас устахаасаа өмнө Казахстанд хүртэл тархаж байсан.
20-р зууны дунд үе хүртэл манай орны Зүүн гарын, Алтайн өвөр говь, Их нууруудын хотгор, Нууруудын хөндий, Баянхонгор, Өмнөговь, Дорноговь, Сүхбаатар аймгийн урд талын цөлөрхөг хээр, зүүн тийш Буйр нуур хүртэлх нутагт хулан тархаж байсан талаар Улаан номд дурдсан байдаг. Одоо Зүүн гарын, Алтайн өвөр говь, Өмнөговь, Дорноговийн баруун хэсэг, Дундговийн урд хэсэгт л бий. Монгол хулангийн тархалт 1990-ээд онд эрс хумигдсан. 1994-1997 оны судалгаагаар өмнөд Монголын бүх нутгаар нийт 33000-63000 толгой хулан тархсаныг тооцоолсон. 2003 онд шинэ судалгаагаар өмнөд Монголын 177563 ам км нутагт 20000 толгой хулан байгааг тооцоолсон. Монгол дахь хулангийн тоо толгойг тооцоолсон судалгааны дүнг хянуур авч үзэх хэрэгтэй. Учир нь Монголд судалгааны батлагдсан протокол байдаггүй. Гэсэн ч сүүлийн 70 жилд Монгол хулангийн тархалтын нутаг дэвсгэр 50\%-иар багассан юм.
Нийт тархалтын 67 хувь нь Дорноговь, 20 хувь нь Өмнөговь, 13 хувь нь Ховд, Говь-Алтай, Баянхонгор аймгийн нутагт байна гэж үздэг. Харин сүүлийн 2-3 жилд хулангийн нөөц, байршил эрс хомсдож байгаа юм. Худалдаж ашиг хонжоо олох зорилгоор хуланг агнах явдал нэмэгдсээр байна.Хулгайн ан, малтай бэлчээр булаацалддаг зэргээс шалтгаалж Монгол хулангийн тоо толгой буурсаар байгаа бөгөөд одоо эмзэг зэрэглэлтэй болсон. 1953 оноос хойш Монголд хуланг дархан цаазтай амьтан болгосон. Хулан нь дэлхийд төдийгүй Монгол орны нэн ховор амьтны болон Зэрлэг амьтан, ургамлын ховордсон зүйлийг олон улсын хэмжээнд худалдаалах тухай конвенц /CITES/-ийн нэгдүгээр хавсралтад бүртгэгдсэн агнахыг хуулиар хориглосон амьтан юм.\par

ажилгүйдлийг
ажилд
баярлалаа
баярын
биеийг

\stopcomparepatterns

\stoptext
