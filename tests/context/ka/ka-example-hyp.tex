%D \module
%D   [      file=s-lan-04,
%D        version=2011.05.01,
%D          title=\CONTEXT\ Style File,
%D       subtitle=Language Environment 4,
%D         author=Hans Hagen,
%D           date=\currentdate,
%D      copyright={PRAGMA / Hans Hagen \& Ton Otten}]
%C
%C This module is part of the \CONTEXT\ macro||package and is
%C therefore copyrighted by \PRAGMA. See mreadme.pdf for
%C details.

%D For Mojca.

% \enabletrackers[fonts.goodies,languages.patterns]

% TODO: lm, lo, mn

\startluacode
    local a_colormodel   = attributes.private('colormodel')

    local nodecodes      = nodes.nodecodes
    local nodepool       = nodes.pool
    local disc           = nodecodes.disc
    local glyph          = nodecodes.glyph
    local quads          = fonts.hashes.quads
    local newkern        = nodepool.kern
    local newrule        = nodepool.rule
    local newglue        = nodepool.glue

    local get_attribute  = node.has_attribute

    local tracers        = nodes.tracers
    local colortracers   = tracers and tracers.colors
    local setnodecolor   = colortracers.set

    local function identify(head,marked)
        local current, prev = head, nil
        while current do
            local id = current.id
            local next = current.next
            if id == disc then
                if prev and next.id == glyph then -- catch other usage of disc
                    marked[#marked+1] = prev
                end
            elseif id == glyph then
                prev = current
            end
            current = next
        end
    end

    local function strip(head,marked)
        for i=1,#marked do
            local prev = marked[i]
            nodes.remove(head,prev.next,true)
        end
    end

    local function mark(head,marked,what,how)
        -- local xquad = quads[marked[1].font]
        head, prev = node.insert_after(head,prev,newkern(1000))
        for i=1,#marked do
            local prev = marked[i]
            local font = prev.font
            local quad = quads[font]
            head, prev = node.insert_after(head,prev,newkern(-quad/20))
            -- head, prev = node.insert_after(head,prev,newkern(0))
            if what then
--                local rule = newrule(quad/10,.75*quad,-0.3*quad)
                local rule = newrule(quad/10,.15*quad,0*quad)
                head, prev = node.insert_after(head,prev,rule)
                setnodecolor(rule,how,get_attribute(prev,a_colormodel))
            else
--                local rule = newrule(quad/10,.15*quad,.3*quad)
                local rule = newrule(quad/10,.75*quad,0*quad)
                head, prev = node.insert_after(head,prev,rule)
                setnodecolor(rule,how,get_attribute(prev,a_colormodel))
            end
            head, prev = node.insert_after(head,prev,newkern(-quad/20))
--            head, prev = node.insert_after(head,prev,newkern(0))
            head, prev = node.insert_after(head,prev,newglue(0))
        end
--        head, prev = node.insert_after(head,prev,newkern(quad/40))
    end

    local language_one, language_two, tag_one, tag_two

    function nodes.showhyphens(head)
        local marked_1, marked_2 = { }, { }
        if language_one then
            for n in node.traverse_id(glyph,head) do
                n.lang = language_one
            end
            lang.hyphenate(head,node.slide(head))
            identify(head,marked_1)
            strip(head,marked_1)
        end
        if language_two then
            for n in node.traverse_id(glyph,head) do
                n.lang = language_two
            end
            lang.hyphenate(head,node.slide(head))
            identify(head,marked_2)
            strip(head,marked_2)
        end
        mark(head,marked_1,true,"hyphenation:"..tag_one)
        if not tag_2 ~= nil then
            mark(head,marked_2,false,"hyphenation:"..tag_two)
        end
        if language_one then
            for n in node.traverse_id(glyph,head) do
                n.lang = language_one
            end
            lang.hyphenate(head,node.slide(head))
        end
        return head, true
    end

    function languages.startcomparepatterns(list)
        local list = utilities.parsers.settings_to_array(list)
        tag_one = list[1]
        tag_two = list[2]
        language_one = tag_one and languages.getnumber(tag_one)
        language_two = tag_two and languages.getnumber(tag_two)
    --  nodes.tasks.appendaction("processors","words","nodes.showhyphens")
        nodes.tasks.appendaction("processors","before","nodes.showhyphens")
        nodes.tasks.enableaction("processors","nodes.showhyphens")
    end

    function languages.stopcomparepatterns()
        nodes.tasks.disableaction("processors","nodes.showhyphens")
    end

\stopluacode

\unprotect

\def\startcomparepatterns[#1]%
  {\bgroup
   \language\zerocount
   \setupalign[\v!nothyphenated]%
   \ctxlua{languages.startcomparepatterns("#1")}}

\def\stopcomparepatterns
  {\ctxlua{languages.stopcomparepatterns()}%
   \egroup}

\protect

\definecolor[hyphenation:ka]  [b=1.]
\definecolor[hyphenation:ka2] [r=.8]
\definecolor[hyphenation:en]  [g=.8]
% \definecolor[hyphenation:nl] [b=.8]

% \doifnotmode{demo}{\endinput}

\installlanguage
	[ka]
	[patterns=ka,
	 lefthyphenchar=-1,
	 righthyphenchar=-1,
	]
\installlanguage
	[ka2]
	[patterns=ka, % use different patterns
	 lefthyphenchar=-1,
	 righthyphenchar=-1,
	]
\mainlanguage
	[ka]

\starttypescript[serif][arial]
	\definefontsynonym [Serif] [file:Arial Unicode.ttf] [features=default]
\stoptypescript

\starttypescript[arial]
	\definetypeface [arial] [rm] [serif] [arial] [default]
\stoptypescript


\setupbodyfont[arial]

\setupwhitespace
	[big]

\definecolor[colorscheme:3:1][r=.8,b=.8]
\definecolor[colorscheme:3:2][b=1.]
\definecolor[colorscheme:3:3][b=.8]
\definecolor[colorscheme:3:4][b=.8]
\definecolor[colorscheme:3:5][b=.4]
\definecolor[colorscheme:3:6][b=.3]
\definecolor[colorscheme:3:7][r=.5]
\definecolor[colorscheme:3:8][g=.5]

\starttext

\setfontcolorscheme[3]

\startcomparepatterns[ka,ka]
    \input ka-text-plain.txt\par
\stopcomparepatterns

% Reference syllabification:
% 
% \catcode`\​=\active
% \def​{\color[red]{\kern-0.5pt\vrule width1pt height 10pt depth0pt\kern-0.5pt}\hskip0pt}
% \input ka-text-hyphenated.txt\par

\stoptext
