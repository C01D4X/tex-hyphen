\documentclass[a4paper,12pt]{article}
\usepackage{polyglossia}
\usepackage{fontspec}
\setdefaultlanguage{sanskrit}
\newfontfamily\sanskritfont[Ligatures=Rare]{Gujarati MT} % or Code2000
\renewcommand{\baselinestretch}{1.1}
\setlength{\textwidth}{7cm}

\begin{document}

% Mahābhārata 1,3,1-195

% 1,3,1
જનમેજયઃ પારિક્ષિતઃ સહ ભ્રાતૃભિઃ કુરુક્ષેત્રે દીર્ઘસત્રમુપાસ્તે। તસ્ય ભ્રાતરસ્ત્રયઃ શ્રુતસેન ઉગ્રસેનો ભીમસેન ઇતિ।
% 1,3,2
તેષુ તત્સત્રમુપાસીનેષુ તત્ર શ્વાભ્યાગચ્છત્સારમેયઃ। સ જનમેજયસ્ય ભ્રાતૃભિરભિહતો રોરૂયમાણો માતુઃ સમીપમુપાગચ્છત્।
% 1,3,3
તં માતા રોરૂયમાણમુવાચ। કિં રોદિષિ। કેનાસ્યભિહત ઇતિ।
% 1,3,4
સ એવમુક્તો માતરં પ્રત્યુવાચ। જનમેજયસ્ય ભ્રાતૃભિરભિહતોઽસ્મીતિ।
% 1,3,5
તં માતા પ્રત્યુવાચ। વ્યક્તં ત્વયા તત્રાપરાદ્ધં યેનાસ્યભિહત ઇતિ।
% 1,3,6
સ તાં પુનરુવાચ। નાપરાધ્યામિ કિંચિત્। નાવેક્ષે હવીંષિ નાવલિહ ઇતિ।
% 1,3,7
તચ્છ્રુત્વા તસ્ય માતા સરમા પુત્રશોકાર્તા તત્સત્રમુપાગચ્છદ્યત્ર સ જનમેજયઃ સહ ભ્રાતૃભિર્દીર્ઘસત્રમુપાસ્તે।
% 1,3,8
સ તયા ક્રુદ્ધયા તત્રોક્તઃ। અયં મે પુત્રો ન કિંચિદપરાધ્યતિ। કિમર્થમભિહત ઇતિ। યસ્માચ્ચાયમભિહતોઽનપકારી તસ્માદદૃષ્ટં ત્વાં ભયમાગમિષ્યતીતિ।
% 1,3,9
સ જનમેજય એવમુક્તો દેવશુન્યા સરમયા દૃઢં સંભ્રાન્તો વિષણ્ણશ્ચાસીત્।
% 1,3,10
સ તસ્મિન્સત્રે સમાપ્તે હાસ્તિનપુરં પ્રત્યેત્ય પુરોહિતમનુરૂપમન્વિચ્છમાનઃ પરં યત્નમકરોદ્યો મે પાપકૃત્યાં શમયેદિતિ।
% 1,3,11
સ કદાચિન્મૃગયાં યાતઃ પારિક્ષિતો જનમેજયઃ કસ્મિંશ્ચિત્સ્વવિષયોદ્દેશે આશ્રમમપશ્યત્।
% 1,3,12
તત્ર કશ્ચિદૃષિરાસાં ચક્રે શ્રુતશ્રવા નામ। તસ્યાભિમતઃ પુત્ર આસ્તે સોમશ્રવા નામ।
% 1,3,13
તસ્ય તં પુત્રમભિગમ્ય જનમેજયઃ પારિક્ષિતઃ પૌરોહિત્યાય વવ્રે।
% 1,3,14
સ નમસ્કૃત્ય તમૃષિમુવાચ। ભગવન્નયં તવ પુત્રો મમ પુરોહિતોઽસ્ત્વિતિ।
% 1,3,15
સ એવમુક્તઃ પ્રત્યુવાચ। ભો જનમેજય પુત્રોઽયં મમ સર્પ્યાં જાતઃ। મહાતપસ્વી સ્વાધ્યાયસંપન્નો મત્તપોવીર્યસંભૃતો મચ્છુક્રં પીતવત્યાસ્તસ્યાઃ કુક્ષૌ સંવૃદ્ધઃ। સમર્થોઽયં ભવતઃ સર્વાઃ પાપકૃત્યાઃ શમયિતુમન્તરેણ મહાદેવકૃત્યામ્। અસ્ય ત્વેકમુપાંશુવ્રતમ્। યદેનં કશ્ચિદ્બ્રાહ્મણઃ કંચિદર્થમભિયાચેત્તં તસ્મૈ દદ્યાદયમ્। યદ્યેતદુત્સહસે તતો નયસ્વૈનમિતિ।
% 1,3,16
તેનૈવમુત્કો જનમેજયસ્તં પ્રત્યુવાચ। ભગવંસ્તથા ભવિષ્યતીતિ।
% 1,3,17
સ તં પુરોહિતમુપાદાયોપાવૃત્તો ભ્રાતૄનુવાચ। મયાયં વૃત ઉપાધ્યાયઃ। યદયં બ્રૂયાત્તત્કાર્યમવિચારયદ્ભિરિતિ।
% 1,3,18
તેનૈવમુક્તા ભ્રાતરસ્તસ્ય તથા ચક્રુઃ। સ તથા ભ્રાતૄન્સંદિશ્ય તક્ષશિલાં પ્રત્યભિપ્રતસ્થે। તં ચ દેશં વશે સ્થાપયામાસ।
% 1,3,19
એતસ્મિન્નન્તરે કશ્ચિદૃષિર્ધૌમ્યો નામાયોદઃ। તસ્ય શિષ્યાસ્ત્રયો બભૂવુરુપમન્યુરારુણિર્વેદશ્ચેતિ।
% 1,3,20
સ એકં શિષ્યમારુણિં પાઞ્ચાલ્યં પ્રેષયામાસ। ગચ્છ કેદારખણ્ડં બધાનેતિ।
% 1,3,21
સ ઉપાધ્યાયેન સંદિષ્ટ આરુણિઃ પાઞ્ચાલ્યસ્તત્ર ગત્વા તત્કેદારખણ્ડં બદ્ધું નાશક્નોત્।
% 1,3,22
સ ક્લિશ્યમાનોઽપશ્યદુપાયમ્। ભવત્વેવં કરિષ્યામીતિ।
% 1,3,23
સ તત્ર સંવિવેશ કેદારખણ્ડે। શયાને તસ્મિંસ્તદુદકં તસ્થૌ।
% 1,3,24
તતઃ કદાચિદુપાધ્યાય આયોદો ધૌમ્યઃ શિષ્યાનપૃચ્છત્। ક્વ આરુણિઃ પાઞ્ચાલ્યો ગત ઇતિ।
% 1,3,25
તે પ્રત્યૂચુઃ। ભગવતૈવ પ્રેષિતો ગચ્છ કેદારખણ્ડં બધાનેતિ।
% 1,3,26
સ એવમુક્તસ્તાઞ્શિષ્યાન્પ્રત્યુવાચ। તસ્માત્સર્વે તત્ર ગચ્છામો યત્ર સ ઇતિ।
% 1,3,27
સ તત્ર ગત્વા તસ્યાહ્વાનાય શબ્દં ચકાર। ભો આરુણે પાઞ્ચાલ્ય ક્વાસિ। વત્સૈહીતિ।
% 1,3,28
સ તચ્છ્રુત્વા આરુણિરુપાધ્યાયવાક્યં તસ્માત્કેદારખણ્ડાત્સહસોત્થાય તમુપાધ્યાયમુપતસ્થે। પ્રોવાચ ચૈનમ્। અયમસ્મ્યત્ર કેદારખણ્ડે નિઃસરમાણમુદકમવારણીયં સંરોદ્ધું સંવિષ્ટો ભગવચ્છબ્દં શ્રુત્વૈવ સહસા વિદાર્ય કેદારખણ્ડં ભવન્તમુપસ્થિતઃ। તદભિવાદયે ભગવન્તમ્। આજ્ઞાપયતુ ભવાન્। કિં કરવાણીતિ।
% 1,3,29
તમુપાધ્યાયોઽબ્રવીત્। યસ્માદ્ભવાન્કેદારખણ્ડમવદાર્યોત્થિતસ્તસ્માદ્ભવાનુદ્દાલક એવ નામ્ના ભવિષ્યતીતિ।
% 1,3,30
સ ઉપાધ્યાયેનાનુગૃહીતઃ। યસ્માત્ત્વયા મદ્વચોઽનુષ્ઠિતં તસ્માચ્છ્રેયોઽવાપ્સ્યસીતિ। સર્વે ચ તે વેદાઃ પ્રતિભાસ્યન્તિ સર્વાણિ ચ ધર્મશાસ્ત્રાણીતિ।
% 1,3,31
સ એવમુક્ત ઉપાધ્યાયેનેષ્ટં દેશં જગામ।
% 1,3,32
અથાપરઃ શિષ્યસ્તસ્યૈવાયોદસ્ય ધૌમ્યસ્યોપમન્યુર્નામ।
% 1,3,33
તમુપાધ્યાયઃ પ્રેષયામાસ। વત્સોપમન્યો ગા રક્ષસ્વેતિ।
% 1,3,34
સ ઉપાધ્યાયવચનાદરક્ષદ્ગાઃ। સ ચાહનિ ગા રક્ષિત્વા દિવસક્ષયેઽભ્યાગમ્યોપાધ્યાયસ્યાગ્રતઃ સ્થિત્વા નમશ્ચક્રે।
% 1,3,35
તમુપાધ્યાયઃ પીવાનમપશ્યત્। ઉવાચ ચૈનમ્। વત્સોપમન્યો કેન વૃત્તિં કલ્પયસિ। પીવાનસિ દૃઢમિતિ।
% 1,3,36
સ ઉપાધ્યાયં પ્રત્યુવાચ। ભૈક્ષેણ વૃત્તિં કલ્પયામીતિ।
% 1,3,37
તમુપાધ્યાયઃ પ્રત્યુવાચ। મમાનિવેદ્ય ભૈક્ષં નોપયોક્તવ્યમિતિ।
% 1,3,38
સ તથેત્યુક્ત્વા પુનરરક્ષદ્ગાઃ। રક્ષિત્વા ચાગમ્ય તથૈવોપાધ્યાયસ્યાગ્રતઃ સ્થિત્વા નમશ્ચક્રે।
% 1,3,39
તમુપાધ્યાયસ્તથાપિ પીવાનમેવ દૃષ્ટ્વોવાચ। વત્સોપમન્યો સર્વમશેષતસ્તે ભૈક્ષં ગૃહ્ણામિ। કેનેદાનીં વૃત્તિં કલ્પયસીતિ।
% 1,3,40
સ એવમુક્ત ઉપાધ્યાયેન પ્રત્યુવાચ। ભગવતે નિવેદ્ય પૂર્વમપરં ચરામિ। તેન વૃત્તિં કલ્પયામીતિ।
% 1,3,41
તમુપાધ્યાયઃ પ્રત્યુવાચ। નૈષા ન્યાય્યા ગુરુવૃત્તિઃ। અન્યેષામપિ વૃત્ત્યુપરોધં કરોષ્યેવં વર્તમાનઃ। લુબ્ધોઽસીતિ।
% 1,3,42
સ તથેત્યુક્ત્વા ગા અરક્ષત્। રક્ષિત્વા ચ પુનરુપાધ્યાયગૃહમાગમ્યોપાધ્યાયસ્યાગ્રતઃ સ્થિત્વા નમશ્ચક્રે।
% 1,3,43
તમુપાધ્યાયસ્તથાપિ પીવાનમેવ દૃષ્ટ્વા પુનરુવાચ। અહં તે સર્વં ભૈક્ષં ગૃહ્ણામિ ન ચાન્યચ્ચરસિ। પીવાનસિ। કેન વૃત્તિં કલ્પયસીતિ।
% 1,3,44
સ ઉપાધ્યાયં પ્રત્યુવાચ। ભો એતાસાં ગવાં પયસા વૃત્તિં કલ્પયામીતિ।
% 1,3,45
તમુપાધ્યાયઃ પ્રત્યુવાચ। નૈતન્ન્યાય્યં પય ઉપયોક્તું ભવતો મયાનનુજ્ઞાતમિતિ।
% 1,3,46
સ તથેતિ પ્રતિજ્ઞાય ગા રક્ષિત્વા પુનરુપાધ્યાયગૃહાનેત્ય ગુરોરગ્રતઃ સ્થિત્વા નમશ્ચક્રે।
% 1,3,47
તમુપાધ્યાયઃ પીવાનમેવાપશ્યત્। ઉવાચ ચૈનમ્। ભૈક્ષં નાશ્નાસિ ન ચાન્યચ્ચરસિ। પયો ન પિબસિ। પીવાનસિ। કેન વૃત્તિં કલ્પયસીતિ।
% 1,3,48
સ એવમુક્ત ઉપાધ્યાયં પ્રત્યુવાચ। ભોઃ ફેનં પિબામિ યમિમે વત્સા માતૄણાં સ્તનં પિબન્ત ઉદ્ગિરન્તીતિ।
% 1,3,49
તમુપાધ્યાયઃ પ્રત્યુવાચ। એતે ત્વદનુકમ્પયા ગુણવન્તો વત્સાઃ પ્રભૂતતરં ફેનમુદ્ગિરન્તિ। તદેવમપિ વત્સાનાં વૃત્ત્યુપરોધં કરોષ્યેવં વર્તમાનઃ। ફેનમપિ ભવાન્ન પાતુમર્હતીતિ।
% 1,3,50
સ તથેતિ પ્રતિજ્ઞાય નિરાહારસ્તા ગા અરક્ષત્। તથા પ્રતિષિદ્ધો ભૈક્ષં નાશ્નાતિ ન ચાન્યચ્ચરતિ। પયો ન પિબતિ। ફેનં નોપયુઙ્ક્તે।
% 1,3,51
સ કદાચિદરણ્યે ક્ષુધાર્તોઽર્કપત્રાણ્યભક્ષયત્।
% 1,3,52
સ તૈરર્કપત્રૈર્ભક્ષિતૈઃ ક્ષારકટૂષ્ણવિપાકિભિશ્ચક્ષુષ્યુપહતોઽન્ધોઽભવત્। સોઽન્ધોઽપિ ચઙ્ક્રમ્યમાણઃ કૂપેઽપતત્।
% 1,3,53
અથ તસ્મિન્નનાગચ્છત્યુપાધ્યાયઃ શિષ્યાનવોચત્। મયોપમન્યુઃ સર્વતઃ પ્રતિષિદ્ધઃ। સ નિયતં કુપિતઃ। તતો નાગચ્છતિ ચિરગતશ્ચેતિ।
% 1,3,54
સ એવમુક્ત્વા ગત્વારણ્યમુપમન્યોરાહ્વાનં ચક્રે। ભો ઉપમન્યો ક્વાસિ। વત્સૈહીતિ।
% 1,3,55
સ તદાહ્વાનમુપાધ્યાયાચ્છ્રુત્વા પ્રત્યુવાચોચ્ચૈઃ। અયમસ્મિ ભો ઉપાધ્યાય કૂપે પતિત ઇતિ।
% 1,3,56
તમુપાધ્યાયઃ પ્રત્યુવાચ। કથમસિ કૂપે પતિત ઇતિ।
% 1,3,57
સ તં પ્રત્યુવાચ। અર્કપત્રાણિ ભક્ષયિત્વાન્ધીભૂતોઽસ્મિ। અતઃ કૂપે પતિત ઇતિ।
% 1,3,58
તમુપાધ્યાયઃ પ્રત્યુવાચ। અશ્વિનૌ સ્તુહિ। તૌ ત્વાં ચક્ષુષ્મન્તં કરિષ્યતો દેવભિષજાવિતિ।
% 1,3,59
સ એવમુક્ત ઉપાધ્યાયેન સ્તોતું પ્રચક્રમે દેવાવશ્વિનૌ વાગ્ભિરૃગ્ભિઃ।
\begin{verse}
% 1,3,60
પ્ર પૂર્વગૌ પૂર્વજૌ ચિત્રભાનૂ ગિરા વા શંસામિ તપનાવનન્તૌ। \\
દિવ્યૌ સુપર્ણૌ વિરજૌ વિમાનાવધિક્ષિયન્તૌ ભુવનાનિ વિશ્વા॥ \\
% 1,3,61
હિરણ્મયૌ શકુની સાંપરાયૌ નાસત્યદસ્રૌ સુનસૌ વૈજયન્તૌ। \\
શુક્રં વયન્તૌ તરસા સુવેમાવભિ વ્યયન્તાવસિતં વિવસ્વત્॥ \\
% 1,3,62
ગ્રસ્તાં સુપર્ણસ્ય બલેન વર્તિકામમુઞ્ચતામશ્વિનૌ સૌભગાય। \\
તાવત્સુવૃત્તાવનમન્ત માયયા સત્તમા ગા અરુણા ઉદાવહન્॥ \\
% 1,3,63
ષષ્ટિશ્ચ ગાવસ્ત્રિશતાશ્ચ ધેનવ એકં વત્સં સુવતે તં દુહન્તિ। \\
નાનાગોષ્ઠા વિહિતા એકદોહનાસ્તાવશ્વિનૌ દુહતો ઘર્મમુક્થ્યમ્॥ \\
% 1,3,64
એકાં નાભિં સપ્તશતા અરાઃ શ્રિતાઃ પ્રધિષ્વન્યા વિંશતિરર્પિતા અરાઃ। \\
અનેમિ ચક્રં પરિવર્તતેઽજરં માયાશ્વિનૌ સમનક્તિ ચર્ષણી॥ \\
% 1,3,65
એકં ચક્રં વર્તતે દ્વાદશારં પ્રધિષણ્ણાભિમેકાક્ષમમૃતસ્ય ધારણમ્। \\
યસ્મિન્દેવા અધિ વિશ્વે વિષક્તાસ્તાવશ્વિનૌ મુઞ્ચતો મા વિષીદતમ્॥ \\
% 1,3,66
અશ્વિનાવિન્દ્રમમૃતં વૃત્તભૂયૌ તિરોધત્તામશ્વિનૌ દાસપત્ની। \\
ભિત્ત્વા ગિરિમશ્વિનૌ ગામુદાચરન્તૌ તદ્વૃષ્ટમહ્ના પ્રથિતા વલસ્ય॥ \\
% 1,3,67
યુવાં દિશો જનયથો દશાગ્રે સમાનં મૂર્ધ્નિ રથયા વિયન્તિ। \\
તાસાં યાતમૃષયોઽનુપ્રયાન્તિ દેવા મનુષ્યાઃ ક્ષિતિમાચરન્તિ॥ \\
% 1,3,68
યુવાં વર્ણાન્વિકુરુથો વિશ્વરૂપાંસ્તેઽધિક્ષિયન્તિ ભુવનાનિ વિશ્વા। \\
તે ભાનવોઽપ્યનુસૃતાશ્ચરન્તિ દેવા મનુષ્યાઃ ક્ષિતિમાચરન્તિ॥ \\
% 1,3,69
તૌ નાસત્યાવશ્વિનાવામહે વાં સ્રજં ચ યાં બિભૃથઃ પુષ્કરસ્ય। \\
તૌ નાસત્યાવમૃતાવૃતાવૃધાવૃતે દેવાસ્તત્પ્રપદેન સૂતે॥ \\
% 1,3,70
મુખેન ગર્ભં લભતાં યુવાનૌ ગતાસુરેતત્પ્રપદેન સૂતે। \\
સદ્યો જાતો માતરમત્તિ ગર્ભસ્તાવશ્વિનૌ મુઞ્ચથો જીવસે ગાઃ॥ \\
\end{verse}
% 1,3,71
એવં તેનાભિષ્ટુતાવશ્વિનાવાજગ્મતુઃ। આહતુશ્ચૈનમ્। પ્રીતૌ સ્વઃ। એષ તેઽપૂપઃ। અશાનૈનમિતિ।
% 1,3,72
સ એવમુક્તઃ પ્રત્યુવાચ। નાનૃતમૂચતુર્ભવન્તૌ। ન ત્વહમેતમપૂપમુપયોક્તુમુત્સહે અનિવેદ્ય ગુરવ ઇતિ।
% 1,3,73
તતસ્તમશ્વિનાવૂચતુઃ। આવાભ્યાં પુરસ્તાદ્ભવત ઉપાધ્યાયેનૈવમેવાભિષ્ટુતાભ્યામપૂપઃ પ્રીતાભ્યાં દત્તઃ। ઉપયુક્તશ્ચ સ તેનાનિવેદ્ય ગુરવે। ત્વમપિ તથૈવ કુરુષ્વ યથા કૃતમુપાધ્યાયેનેતિ।
% 1,3,74
સ એવમુક્તઃ પુનરેવ પ્રત્યુવાચૈતૌ। પ્રત્યનુનયે ભવન્તાવશ્વિનૌ। નોત્સહેઽહમનિવેદ્યોપાધ્યાયાયોપયોક્તુમિતિ।
% 1,3,75
તમશ્વિનાવાહતુઃ। પ્રીતૌ સ્વસ્તવાનયા ગુરુવૃત્ત્યા। ઉપાધ્યાયસ્ય તે કાર્ષ્ણાયસા દન્તાઃ। ભવતો હિરણ્મયા ભવિષ્યન્તિ। ચક્ષુષ્માંશ્ચ ભવિષ્યસિ। શ્રેયશ્ચાવાપ્સ્યસીતિ।
% 1,3,76
સ એવમુક્તોઽશ્વિભ્યાં લબ્ધચક્ષુરુપાધ્યાયસકાશમાગમ્યોપાધ્યાયમભિવાદ્યાચચક્ષે। સ ચાસ્ય પ્રીતિમાનભૂત્।
% 1,3,77
આહ ચૈનમ્। યથાશ્વિનાવાહતુસ્તથા ત્વં શ્રેયોઽવાપ્સ્યસીતિ। સર્વે ચ તે વેદાઃ પ્રતિભાસ્યન્તીતિ।
% 1,3,78
એષા તસ્યાપિ પરીક્ષોપમન્યોઃ।
% 1,3,79
અથાપરઃ શિષ્યસ્તસ્યૈવાયોદસ્ય ધૌમ્યસ્ય વેદો નામ।
% 1,3,80
તમુપાધ્યાયઃ સંદિદેશ। વત્સ વેદ ઇહાસ્યતામ્। ભવતા મદ્ગૃહે કંચિત્કાલં શુશ્રૂષમાણેન ભવિતવ્યમ્। શ્રેયસ્તે ભવિષ્યતીતિ।
% 1,3,81
સ તથેત્યુક્ત્વા ગુરુકુલે દીર્ઘકાલં ગુરુશુશ્રૂષણપરોઽવસત્। ગૌરિવ નિત્યં ગુરુષુ ધૂર્ષુ નિયુજ્યમાનઃ શીતોષ્ણક્ષુત્તૃષ્ણાદુઃખસહઃ સર્વત્રાપ્રતિકૂલઃ।
% 1,3,82
તસ્ય મહતા કાલેન ગુરુઃ પરિતોષં જગામ। તત્પરિતોષાચ્ચ શ્રેયઃ સર્વજ્ઞતાં ચાવાપ। એષા તસ્યાપિ પરીક્ષા વેદસ્ય।
% 1,3,83
સ ઉપાધ્યાયેનાનુજ્ઞાતઃ સમાવૃત્તસ્તસ્માદ્ગુરુકુલવાસાદ્ગૃહાશ્રમં પ્રત્યપદ્યત। તસ્યાપિ સ્વગૃહે વસતસ્ત્રયઃ શિષ્યા બભૂવુઃ।
% 1,3,84
સ શિષ્યાન્ન કિંચિદુવાચ। કર્મ વા ક્રિયતાં ગુરુશુશ્રૂષા વેતિ। દુઃખાભિજ્ઞો હિ ગુરુકુલવાસસ્ય શિષ્યાન્પરિક્લેશેન યોજયિતું નેયેષ।
% 1,3,85
અથ કસ્યચિત્કાલસ્ય વેદં બ્રાહ્મણં જનમેજયઃ પૌષ્યશ્ચ ક્ષત્રિયાવુપેત્યોપાધ્યાયં વરયાં ચક્રતુઃ।
% 1,3,86
સ કદાચિદ્યાજ્યકાર્યેણાભિપ્રસ્થિત ઉત્તઙ્કં નામ શિષ્યં નિયોજયામાસ। ભો ઉત્તઙ્ક યત્કિંચિદસ્મદ્ગૃહે પરિહીયતે તદિચ્છામ્યહમપરિહીણં ભવતા ક્રિયમાણમિતિ।
% 1,3,87
સ એવં પ્રતિસમાદિશ્યોત્તઙ્કં વેદઃ પ્રવાસં જગામ।
% 1,3,88
અથોત્તઙ્કો ગુરુશુશ્રૂષુર્ગુરુનિયોગમનુતિષ્ઠમાનસ્તત્ર ગુરુકુલે વસતિ સ્મ।
% 1,3,89
સ વસંસ્તત્રોપાધ્યાયસ્ત્રીભિઃ સહિતાભિરાહૂયોક્તઃ। ઉપાધ્યાયિની તે ઋતુમતી। ઉપાધ્યાયશ્ચ પ્રોષિતઃ। અસ્યા યથાયમૃતુર્વન્ધ્યો ન ભવતિ તથા ક્રિયતામ્। એતદ્વિષીદતીતિ।
% 1,3,90
સ એવમુક્તસ્તાઃ સ્ત્રિયઃ પ્રત્યુવાચ। ન મયા સ્ત્રીણાં વચનાદિદમકાર્યં કાર્યમ્। ન હ્યહમુપાધ્યાયેન સંદિષ્ટઃ। અકાર્યમપિ ત્વયા કાર્યમિતિ।
% 1,3,91
તસ્ય પુનરુપાધ્યાયઃ કાલાન્તરેણ ગૃહાનુપજગામ તસ્માત્પ્રવાસાત્। સ તદ્વૃત્તં તસ્યાશેષમુપલભ્ય પ્રીતિમાનભૂત્।
% 1,3,92
ઉવાચ ચૈનમ્। વત્સોત્તઙ્ક કિં તે પ્રિયં કરવાણીતિ। ધર્મતો હિ શુશ્રૂષિતોઽસ્મિ ભવતા। તેન પ્રીતિઃ પરસ્પરેણ નૌ સંવૃદ્ધા। તદનુજાને ભવન્તમ્। સર્વામેવ સિદ્ધિં પ્રાપ્સ્યસિ। ગમ્યતામિતિ।
% 1,3,93
સ એવમુક્તઃ પ્રત્યુવાચ। કિં તે પ્રિયં કરવાણીતિ। એવં હ્યાહુઃ।
% 1,3,94
યશ્ચાધર્મેણ વિબ્રૂયાદ્યશ્ચાધર્મેણ પૃચ્છતિ।
તયોરન્યતરઃ પ્રૈતિ વિદ્વેષં ચાધિગચ્છતિ।
% 1,3,95
સોઽહમનુજ્ઞાતો ભવતા ઇચ્છામીષ્ટં તે ગુર્વર્થમુપહર્તુમિતિ।
% 1,3,96
તેનૈવમુક્ત ઉપાધ્યાયઃ પ્રત્યુવાચ। વત્સોત્તઙ્ક ઉષ્યતાં તાવદિતિ।
% 1,3,97
સ કદાચિત્તમુપાધ્યાયમાહોત્તઙ્કઃ। આજ્ઞાપયતુ ભવાન્। કિં તે પ્રિયમુપહરામિ ગુર્વર્થમિતિ।
% 1,3,98
તમુપાધ્યાયઃ પ્રત્યુવાચ। વત્સોત્તઙ્ક બહુશો માં ચોદયસિ ગુર્વર્થમુપહરેયમિતિ। તદ્ગચ્છ। એનાં પ્રવિશ્યોપાધ્યાયિનીં પૃચ્છ કિમુપહરામીતિ। એષા યદ્બ્રવીતિ તદુપહરસ્વેતિ।
% 1,3,99
સ એવમુક્ત ઉપાધ્યાયેનોપાધ્યાયિનીમપૃચ્છત્। ભવત્યુપાધ્યાયેનાસ્મ્યનુજ્ઞાતો ગૃહં ગન્તુમ્। તદિચ્છામીષ્ટં તે ગુર્વર્થમુપહૃત્યાનૃણો ગન્તુમ્। તદાજ્ઞાપયતુ ભવતી। કિમુપહરામિ ગુર્વર્થમિતિ।
% 1,3,100
સૈવમુક્તોપાધ્યાયિન્યુત્તઙ્કં પ્રત્યુવાચ। ગચ્છ પૌષ્યં રાજાનમ્। ભિક્ષસ્વ તસ્ય ક્ષત્રિયયા પિનદ્ધે કુણ્ડલે। તે આનયસ્વ। ઇતશ્ચતુર્થેઽહનિ પુણ્યકં ભવિતા। તાભ્યામાબદ્ધાભ્યાં બ્રાહ્મણાન્પરિવેષ્ટુમિચ્છામિ। શોભમાના યથા તાભ્યાં કુણ્ડલાભ્યાં તસ્મિન્નહનિ સંપાદયસ્વ। શ્રેયો હિ તે સ્યાત્ક્ષણં કુર્વત ઇતિ।
% 1,3,11
સ એવમુક્ત ઉપાધ્યાયિન્યા પ્રાતિષ્ઠતોત્તઙ્કઃ। સ પથિ ગચ્છન્નપશ્યદૃષભમતિપ્રમાણં તમધિરૂઢં ચ પુરુષમતિપ્રમાણમેવ।
% 1,3,102
સ પુરુષ ઉત્તઙ્કમભ્યભાષત। ઉત્તઙ્કૈતત્પુરીષમસ્ય ઋષભસ્ય ભક્ષયસ્વેતિ।
% 1,3,103
સ એવમુક્તો નૈચ્છત્।
% 1,3,104
તમાહ પુરુષો ભૂયઃ। ભક્ષયસ્વોત્તઙ્ક। મા વિચારય। ઉપાધ્યાયેનાપિ તે ભક્ષિતં પૂર્વમિતિ।
% 1,3,105
સ એવમુક્તો બાઢમિત્યુક્ત્વા તદા તદૃષભસ્ય પુરીષં મૂત્રં ચ ભક્ષયિત્વોત્તઙ્કઃ પ્રતસ્થે યત્ર સ ક્ષત્રિયઃ પૌષ્યઃ।
% 1,3,106
તમુપેત્યાપશ્યદુત્તઙ્ક આસીનમ્। સ તમુપેત્યાશીર્ભિરભિનન્દ્યોવાચ। અર્થી ભવન્તમુપગતોઽસ્મીતિ।
% 1,3,107
સ એનમભિવાદ્યોવાચ। ભગવન્પૌષ્યઃ ખલ્વહમ્। કિં કરવાણીતિ।
% 1,3,108
તમુવાચોત્તઙ્કઃ। ગુર્વર્થે કુણ્ડલાભ્યામર્થ્યાગતોઽસ્મીતિ યે તે ક્ષત્રિયયા પિનદ્ધે કુણ્ડલે તે ભવાન્દાતુમર્હતીતિ।
% 1,3,109
તં પૌષ્યઃ પ્રત્યુવાચ। પ્રવિશ્યાન્તઃપુરં ક્ષત્રિયા યાચ્યતામિતિ।
% 1,3,110
સ તેનૈવમુક્તઃ પ્રવિશ્યાન્તઃપુરં ક્ષત્રિયાં નાપશ્યત્।
% 1,3,111
સ પૌષ્યં પુનરુવાચ। ન યુક્તં ભવતા વયમનૃતેનોપચરિતુમ્। ન હિ તે ક્ષત્રિયાન્તઃપુરે સંનિહિતા। નૈનાં પશ્યામીતિ।
% 1,3,112
સ એવમુક્તઃ પૌષ્યસ્તં પ્રત્યુવાચ। સંપ્રતિ ભવાનુચ્છિષ્ટઃ। સ્મર તાવત્। ન હિ સા ક્ષત્રિયા ઉચ્છિષ્ટેનાશુચિના વા શક્યા દ્રષ્ટુમ્। પતિવ્રતાત્વાદેષા નાશુચેર્દર્શનમુપૈતીતિ।
% 1,3,113
અથૈવમુક્ત ઉત્તઙ્કઃ સ્મૃત્વોવાચ। અસ્તિ ખલુ મયોચ્છિષ્ટેનોપસ્પૃષ્ટં શીઘ્રં ગચ્છતા ચેતિ।
% 1,3,114
તં પૌષ્યઃ પ્રત્યુવાચ। એતત્તદેવં હિ। ન ગચ્છતોપસ્પૃષ્ટં ભવતિ ન સ્થિતેનેતિ।
% 1,3,115
અથોત્તઙ્કસ્તથેત્યુક્ત્વા પ્રાઙ્મુખ ઉપવિશ્ય સુપ્રક્ષાલિતપાણિપાદવદનોઽશબ્દાભિર્હૃદયંગમાભિરદ્ભિરુપસ્પૃશ્ય ત્રિઃ પીત્વા દ્વિઃ પરિમૃજ્ય ખાન્યદ્ભિરુપસ્પૃશ્યાન્તઃપુરં પ્રવિશ્ય તાં ક્ષત્રિયામપશ્યત્।
% 1,3,116
સા ચ દૃષ્ટ્વૈવોત્તઙ્કમભ્યુત્થાયાભિવાદ્યોવાચ। સ્વાગતં તે ભગવન્। આજ્ઞાપય કિં કરવાણીતિ।
% 1,3,117
સ તામુવાચ। એતે કુણ્ડલે ગુર્વર્થં મે ભિક્ષિતે દાતુમર્હસીતિ।
% 1,3,118
સા પ્રીતા તેન તસ્ય સદ્ભાવેન પાત્રમયમનતિક્રમણીયશ્ચેતિ મત્વા તે કુણ્ડલે અવમુચ્યાસ્મૈ પ્રાયચ્છત્।
% 1,3,119
આહ ચૈનમ્। એતે કુણ્ડલે તક્ષકો નાગરાજઃ પ્રાર્થયતિ। અપ્રમત્તો નેતુમર્હસીતિ।
% 1,3,120
સ એવમુક્તસ્તાં ક્ષત્રિયાં પ્રત્યુવાચ। ભવતિ સુનિર્વૃતા ભવ। ન માં શક્તસ્તક્ષકો નાગરાજો ધર્ષયિતુમિતિ।
% 1,3,121
સ એવમુક્ત્વા તાં ક્ષત્રિયામામન્ત્ર્ય પૌષ્યસકાશમાગચ્છત્।
% 1,3,122
સ તં દૃષ્ટ્વોવાચ। ભોઃ પૌષ્ય પ્રીતોઽસ્મીતિ।
% 1,3,123
તં પૌષ્યઃ પ્રત્યુવાચ। ભગવંશ્ચિરસ્ય પાત્રમાસાદ્યતે। ભવાંશ્ચ ગુણવાનતિથિઃ। તત્કરિષ્યે શ્રાદ્ધમ્। ક્ષણઃ ક્રિયતામિતિ।
% 1,3,124
તમુત્તઙ્કઃ પ્રત્યુવાચ। કૃતક્ષણ એવાસ્મિ। શીઘ્રમિચ્છામિ યથોપપન્નમન્નમુપહૃતં ભવતેતિ।
% 1,3,125
સ તથેત્યુક્ત્વા યથોપપન્નેનાન્નેનૈનં ભોજયામાસ।
% 1,3,126
અથોત્તઙ્કઃ શીતમન્નં સકેશં દૃષ્ટ્વા અશુચ્યેતદિતિ મત્વા પૌષ્યમુવાચ। યસ્માન્મે અશુચ્યન્નં દદાસિ તસ્મદન્ધો ભવિષ્યસીતિ।
% 1,3,127
તં પૌષ્યઃ પ્રત્યુવાચ। યસ્માત્ત્વમપ્યદુષ્ટમન્નં દૂષયસિ તસ્માદનપત્યો ભવિષ્યસીતિ।
% 1,3,128
સોઽથ પૌષ્યસ્તસ્યાશુચિભાવમન્નસ્યાગમયામાસ।
% 1,3,129
અથ તદન્નં મુક્તકેશ્યા સ્ત્રિયોપહૃતં સકેશમશુચિ મત્વોત્તઙ્કં પ્રસાદયામાસ। ભગવન્નજ્ઞાનાદેતદન્નં સકેશમુપહૃતં શીતં ચ। તત્ક્ષામયે ભવન્તમ્। ન ભવેયમન્ધ ઇતિ।
% 1,3,130
તમુત્તઙ્કઃ પ્રત્યુવાચ। ન મૃષા બ્રવીમિ। ભૂત્વા ત્વમન્ધો નચિરાદનન્ધો ભવિષ્યસીતિ। મમાપિ શાપો ન ભવેદ્ભવતા દત્ત ઇતિ।
% 1,3,131
તં પૌષ્યઃ પ્રત્યુવાચ। નાહં શક્તઃ શાપં પ્રત્યાદાતુમ્। ન હિ મે મન્યુરદ્યાપ્યુપશમં ગચ્છતિ। કિં ચૈતદ્ભવતા ન જ્ઞાયતે યથા।
\begin{verse}
% 1,3,132
નાવનીતં હૃદયં બ્રાહ્મણસ્ય વાચિ ક્ષુરો નિહિતસ્તીક્ષ્ણધારઃ। \\
વિપરીતમેતદુભયં ક્ષત્રિયસ્ય વાઙ્નાવનીતી હૃદયં તીક્ષ્ણધારમ્॥ \\
\end{verse}
% 1,3,133
ઇતિ। તદેવં ગતે ન શક્તોઽહં તીક્ષ્ણહૃદયત્વાત્તં શાપમન્યથા કર્તુમ્। ગમ્યતામિતિ।
% 1,3,134
તમુત્તઙ્કઃ પ્રત્યુવાચ। ભવતાહમન્નસ્યાશુચિભાવમાગમય્ય પ્રત્યનુનીતઃ। પ્રાક્ચ તેઽભિહિતમ્। યસ્માદદુષ્ટમન્નં દૂષયસિ તસ્માદનપત્યો ભવિષ્યસીતિ। દુષ્ટે ચાન્ને નૈષ મમ શાપો ભવિષ્યતીતિ।
% 1,3,135
સાધયામસ્તાવદિત્યુક્ત્વા પ્રાતિષ્ઠતોત્તઙ્કસ્તે કુણ્ડલે ગૃહીત્વા।
% 1,3,136
સોઽપશ્યત્પથિ નગ્નં શ્રમણમાગચ્છન્તં મુહુર્મુહુર્દૃશ્યમાનમદૃશ્યમાનં ચ। અથોત્તઙ્કસ્તે કુણ્ડલે ભૂમૌ નિક્ષિપ્યોદકાર્થં પ્રચક્રમે।
% 1,3,137
એતસ્મિન્નન્તરે સ શ્રમણસ્ત્વરમાણ ઉપસૃત્ય તે કુણ્ડલે ગૃહીત્વા પ્રાદ્રવત્। તમુત્તઙ્કોઽભિસૃત્ય જગ્રાહ। સ તદ્રૂપં વિહાય તક્ષકરૂપં કૃત્વા સહસા ધરણ્યાં વિવૃતં મહાબિલં વિવેશ।
% 1,3,138
પ્રવિશ્ય ચ નાગલોકં સ્વભવનમગચ્છત્। તમુત્તઙ્કોઽન્વાવિવેશ તેનૈવ બિલેન। પ્રવિશ્ય ચ નાગાનસ્તુવદેભિઃ શ્લોકૈઃ।
\begin{verse}
% 1,3,139
ય ઐરાવતરાજાનઃ સર્પાઃ સમિતિશોભનાઃ। \\
વર્ષન્ત ઇવ જીમૂતાઃ સવિદ્યુત્પવનેરિતાઃ॥ \\
% 1,3,140
સુરૂપાશ્ચ વિરૂપાશ્ચ તથા કલ્માષકુણ્ડલાઃ। \\
આદિત્યવન્નાકપૃષ્ઠે રેજુરૈરાવતોદ્ભવાઃ॥ \\
% 1,3,141
બહૂનિ નાગવર્ત્માનિ ગઙ્ગાયાસ્તીર ઉત્તરે। \\
ઇચ્છેત્કોઽર્કાંશુસેનાયાં ચર્તુમૈરાવતં વિના॥ \\
% 1,3,142
શતાન્યશીતિરષ્ટૌ ચ સહસ્રાણિ ચ વિંશતિઃ। \\
સર્પાણાં પ્રગ્રહા યાન્તિ ધૃતરાષ્ટ્રો યદેજતિ॥ \\
% 1,3,143
યે ચૈનમુપસર્પન્તિ યે ચ દૂરં પરં ગતાઃ। \\
અહમૈરાવતજ્યેષ્ઠભ્રાતૃભ્યોઽકરવં નમઃ॥ \\
% 1,3,144
યસ્ય વાસઃ કુરુક્ષેત્રે ખાણ્ડવે ચાભવત્સદા। \\
તં કાદ્રવેયમસ્તૌષં કુણ્ડલાર્થાય તક્ષકમ્॥ \\
% 1,3,145
તક્ષકશ્ચાશ્વસેનશ્ચ નિત્યં સહચરાવુભૌ। \\
કુરુક્ષેત્રે નિવસતાં નદીમિક્ષુમતીમનુ॥ \\
% 1,3,146
જઘન્યજસ્તક્ષકસ્ય શ્રુતસેનેતિ યઃ શ્રુતઃ। \\
અવસદ્યો મહદ્દ્યુમ્નિ પ્રાર્થયન્નાગમુખ્યતામ્। \\
કરવાણિ સદા ચાહં નમસ્તસ્મૈ મહાત્મને॥ \\
\end{verse}
% 1,3,147
એવં સ્તુવન્નપિ નાગાન્યદા તે કુણ્ડલે નાલભદથાપશ્યત્સ્ત્રિયૌ તન્ત્રે અધિરોપ્ય પટં વયન્ત્યૌ।
% 1,3,148
તસ્મિંશ્ચ તન્ત્રે કૃષ્ણાઃ સિતાશ્ચ તન્તવઃ। ચક્રં ચાપશ્યત્ષડ્ભિઃ કુમારૈઃ પરિવર્ત્યમાનમ્। પુરુષં ચાપશ્યદ્દર્શનીયમ્।
% 1,3,149
સ તાન્સર્વાંસ્તુષ્ટાવ એભિર્મન્ત્રવાદશ્લોકૈઃ।
\begin{verse}
% 1,3,150
ત્રીણ્યર્પિતાન્યત્ર શતાનિ મધ્યે ષષ્ટિશ્ચ નિત્યં ચરતિ ધ્રુવેઽસ્મિન્। \\
ચક્રે ચતુર્વિંશતિપર્વયોગે ષડ્યત્કુમારાઃ પરિવર્તયન્તિ॥ \\
% 1,3,151
તન્ત્રં ચેદં વિશ્વરૂપં યુવત્યૌ વયતસ્તન્તૂન્સતતં વર્તયન્ત્યૌ। \\
કૃષ્ણાન્સિતાંશ્ચૈવ વિવર્તયન્ત્યૌ ભૂતાન્યજસ્રં ભુવનાનિ ચૈવ॥ \\
% 1,3,152
વજ્રસ્ય ભર્તા ભુવનસ્ય ગોપ્તા વૃત્રસ્ય હન્તા નમુચેર્નિહન્તા। \\
કૃષ્ણે વસાનો વસને મહાત્મા સત્યાનૃતે યો વિવિનક્તિ લોકે॥ \\
% 1,3,153
યો વાજિનં ગર્ભમપાં પુરાણં વૈશ્વાનરં વાહનમભ્યુપેતઃ। \\
નમઃ સદાસ્મૈ જગદીશ્વરાય લોકત્રયેશાય પુરંદરાય॥ \\
\end{verse}
% 1,3,154
તતઃ સ એનં પુરુષઃ પ્રાહ। પ્રીતોઽસ્મિ તેઽહમનેન સ્તોત્રેણ। કિં તે પ્રિયં કરવાણીતિ।
% 1,3,155
સ તમુવાચ। નાગા મે વશમીયુરિતિ।
% 1,3,156
સ એનં પુરુષઃ પુનરુવાચ। એતમશ્વમપાને ધમસ્વેતિ।
% 1,3,157
સ તમશ્વમપાનેઽધમત્। અથાશ્વાદ્ધમ્યમાનાત્સર્વસ્રોતોભ્યઃ સધૂમા અર્ચિષોઽગ્નેર્નિષ્પેતુઃ।
% 1,3,158
તાભિર્નાગલોકો ધૂપિતઃ।
% 1,3,159
અથ સસંભ્રમસ્તક્ષકોઽગ્નિતેજોભયવિષણ્ણસ્તે કુણ્ડલે ગૃહીત્વા સહસા સ્વભવનાન્નિષ્ક્રમ્યોત્તઙ્કમુવાચ। એતે કુણ્ડલે પ્રતિગૃહ્ણાતુ ભવાનિતિ।
% 1,3,160
સ તે પ્રતિજગ્રાહોત્તઙ્કઃ। કુણ્ડલે પ્રતિગૃહ્યાચિન્તયત્। અદ્ય તત્પુણ્યકમુપાધ્યાયિન્યાઃ। દૂરં ચાહમભ્યાગતઃ। કથં નુ ખલુ સંભાવયેયમિતિ।
% 1,3,161
તત એનં ચિન્તયાનમેવ સ પુરુષ ઉવાચ। ઉત્તઙ્ક એનમશ્વમધિરોહ। એષ ત્વાં ક્ષણાદેવોપાધ્યાયકુલં પ્રાપયિષ્યતીતિ।
% 1,3,162
સ તથેત્યુક્ત્વા તમશ્વમધિરુહ્ય પ્રત્યાજગામોપાધ્યાયકુલમ્। ઉપાધ્યાયિની ચ સ્નાતા કેશાનાવપયન્ત્યુપવિષ્ટોત્તઙ્કો નાગચ્છતીતિ શાપાયાસ્ય મનો દધે।
% 1,3,163
અથોત્તઙ્કઃ પ્રવિશ્ય ઉપાધ્યાયિનીમભ્યવાદયત્। તે ચાસ્યૈ કુણ્ડલે પ્રાયચ્છત્।
% 1,3,164
સા ચૈનં પ્રત્યુવાચ। ઉત્તઙ્ક દેશે કાલેઽભ્યાગતઃ। સ્વાગતં તે વત્સ। મનાગસિ મયા ન શપ્તઃ। શ્રેયસ્તવોપસ્થિતમ્। સિદ્ધિમાપ્નુહીતિ।
% 1,3,165
અથોત્તઙ્ક ઉપાધ્યાયમભ્યવાદયત્। તમુપાધ્યાયઃ પ્રત્યુવાચ। વત્સોત્તઙ્ક સ્વાગતં તે। કિં ચિરં કૃતમિતિ।
% 1,3,166
તમુત્તઙ્ક ઉપાધ્યાયં પ્રત્યુવાચ। ભોસ્તક્ષકેણ નાગરાજેન વિઘ્નઃ કૃતોઽસ્મિન્કર્મણિ। તેનાસ્મિ નાગલોકં નીતઃ।
% 1,3,167
તત્ર ચ મયા દૃષ્ટે સ્ત્રિયૌ તન્ત્રેઽધિરોપ્ય પટં વયન્ત્યૌ। તસ્મિંશ્ચ તન્ત્રે કૃષ્ણાઃ સિતાશ્ચ તન્તવઃ। કિં તત્।
% 1,3,168
તત્ર ચ મયા ચક્રં દૃષ્ટં દ્વાદશારમ્। ષટ્ચૈનં કુમારાઃ પરિવર્તયન્તિ। તદપિ કિમ્।
% 1,3,169
પુરુષશ્ચાપિ મયા દૃષ્ટઃ। સ પુનઃ કઃ।
% 1,3,170
અશ્વશ્ચાતિપ્રમાણયુક્તઃ। સ ચાપિ કઃ।
% 1,3,171
પથિ ગચ્છતા મયા ઋષભો દૃષ્ટઃ। તં ચ પુરુષોઽધિરૂઢઃ। તેનાસ્મિ સોપચારમુક્તઃ। ઉત્તઙ્કાસ્ય ઋષભસ્ય પુરીષં ભક્ષય। ઉપાધ્યાયેનાપિ તે ભક્ષિતમિતિ। તતસ્તદ્વચનાન્મયા તદૃષભસ્ય પુરીષમુપયુક્તમ્। તદિચ્છામિ ભવતોપદિષ્ટં કિં તદિતિ।
% 1,3,172
તેનૈવમુક્ત ઉપાધ્યાયઃ પ્રત્યુવાચ। યે તે સ્ત્રિયૌ ધાતા વિધાતા ચ। યે ચ તે કૃષ્ણાઃ સિતાશ્ચ તન્તવસ્તે રાત્ર્યહની।
% 1,3,173
યદપિ તચ્ચક્રં દ્વાદશારં ષટ્કુમારાઃ પરિવર્તયન્તિ તે ઋતવઃ ષટ્સંવત્સરશ્ચક્રમ્। યઃ પુરુષઃ સ પર્જન્યઃ। યોઽશ્વઃ સોઽગ્નિઃ।
% 1,3,174
ય ઋષભસ્ત્વયા પથિ ગચ્છતા દૃષ્ટઃ સ ઐરાવતો નાગરાજઃ। યશ્ચૈનમધિરૂઢઃ સ ઇન્દ્રઃ। યદપિ તે પુરીષં ભક્ષિતં તસ્ય ઋષભસ્ય તદમૃતમ્।
% 1,3,175
તેન ખલ્વસિ ન વ્યાપન્નસ્તસ્મિન્નાગભવને। સ ચાપિ મમ સખા ઇન્દ્રઃ।
% 1,3,176
તદનુગ્રહાત્કુણ્ડલે ગૃહીત્વા પુનરભ્યાગતોઽસિ। તત્સૌમ્ય ગમ્યતામ્। અનુજાને ભવન્તમ્। શ્રેયોઽવાપ્સ્યસીતિ।
% 1,3,177
સ ઉપાધ્યાયેનાનુજ્ઞાત ઉત્તઙ્કઃ ક્રુદ્ધસ્તક્ષકસ્ય પ્રતિચિકીર્ષમાણો હાસ્તિનપુરં પ્રતસ્થે।
\begin{verse}
% 1,3,178
સ હાસ્તિનપુરં પ્રાપ્ય નચિરાદ્દ્વિજસત્તમઃ। \\
સમાગચ્છત રાજાનમુત્તઙ્કો જનમેજયમ્॥ \\
% 1,3,179
પુરા તક્ષશિલાતસ્તં નિવૃત્તમપરાજિતમ્। \\
સમ્યગ્વિજયિનં દૃષ્ટ્વા સમન્તાન્મન્ત્રિભિર્વૃતમ્॥ \\
% 1,3,180
તસ્મૈ જયાશિષઃ પૂર્વં યથાન્યાયં પ્રયુજ્ય સઃ। \\
ઉવાચૈનં વચઃ કાલે શબ્દસંપન્નયા ગિરા॥ \\
% 1,3,181
અન્યસ્મિન્કરણીયે ત્વં કાર્યે પાર્થિવસત્તમ। \\
બાલ્યાદિવાન્યદેવ ત્વં કુરુષે નૃપસત્તમ॥ \\
% 1,3,182
એવમુક્તસ્તુ વિપ્રેણ સ રાજા પ્રત્યુવાચ હ। \\
જનમેજયઃ પ્રસન્નાત્મા સમ્યક્સંપૂજ્ય તં મુનિમ્॥ \\
% 1,3,183
આસાં પ્રજાનાં પરિપાલનેન સ્વં ક્ષત્રધર્મં પરિપાલયામિ। \\
પ્રબ્રૂહિ વા કિં ક્રિયતાં દ્વિજેન્દ્ર શુશ્રૂષુરસ્મ્યદ્ય વચસ્ત્વદીયમ્॥ \\
% 1,3,184
સ એવમુક્તસ્તુ નૃપોત્તમેન દ્વિજોત્તમઃ પુણ્યકૃતાં વરિષ્ઠઃ। \\
ઉવાચ રાજાનમદીનસત્ત્વં સ્વમેવ કાર્યં નૃપતેશ્ચ યત્તત્॥ \\
% 1,3,185
તક્ષકેણ નરેન્દ્રેન્દ્ર યેન તે હિંસિતઃ પિતા। \\
તસ્મૈ પ્રતિકુરુષ્વ ત્વં પન્નગાય દુરાત્મને॥ \\
% 1,3,186
કાર્યકાલં ચ મન્યેઽહં વિધિદૃષ્ટસ્ય કર્મણઃ। \\
તદ્ગચ્છાપચિતિં રાજન્પિતુસ્તસ્ય મહાત્મનઃ॥ \\
% 1,3,187
તેન હ્યનપરાધી સ દષ્ટો દુષ્ટાન્તરાત્મના। \\
પઞ્ચત્વમગમદ્રાજા વજ્રાહત ઇવ દ્રુમઃ॥ \\
% 1,3,188
બલદર્પસમુત્સિક્તસ્તક્ષકઃ પન્નગાધમઃ। \\
અકાર્યં કૃતવાન્પાપો યોઽદશત્પિતરં તવ॥ \\
% 1,3,189
રાજર્ષિવંશગોપ્તારમમરપ્રતિમં નૃપમ્। \\
જઘાન કાશ્યપં ચૈવ ન્યવર્તયત પાપકૃત્॥ \\
% 1,3,190
દગ્ધુમર્હસિ તં પાપં જ્વલિતે હવ્યવાહને। \\
સર્પસત્રે મહારાજ ત્વયિ તદ્ધિ વિધીયતે॥ \\
% 1,3,191
એવં પિતુશ્ચાપચિતિં ગતવાંસ્ત્વં ભવિષ્યસિ। \\
મમ પ્રિયં ચ સુમહત્કૃતં રાજન્ભવિષ્યતિ॥ \\
% 1,3,192
કર્મણઃ પૃથિવીપાલ મમ યેન દુરાત્મના। \\
વિઘ્નઃ કૃતો મહારાજ ગુર્વર્થં ચરતોઽનઘ॥ \\
% 1,3,193
એતચ્છ્રુત્વા તુ નૃપતિસ્તક્ષકસ્ય ચુકોપ હ। \\
ઉત્તઙ્કવાક્યહવિષા દીપ્તોઽગ્નિર્હવિષા યથા॥ \\
% 1,3,194
અપૃચ્છચ્ચ તદા રાજા મન્ત્રિણઃ સ્વાન્સુદુઃખિતઃ। \\
ઉત્તઙ્કસ્યૈવ સાંનિધ્યે પિતુઃ સ્વર્ગગતિં પ્રતિ॥ \\
% 1,3,195
તદૈવ હિ સ રાજેન્દ્રો દુઃખશોકાપ્લુતોઽભવત્। \\
યદૈવ પિતરં વૃત્તમુત્તઙ્કાદશૃણોત્તદા॥ \\
\end{verse}

\end{document}

