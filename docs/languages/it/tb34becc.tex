% Computer Aided Hyphenation for Italian and Modern Latin
% by Claudio Beccari
% Dipartimento di Elettronica
% Politecnico di Torino
% e-mail beccari@polito.it
%
\documentstyle[ltugboat]{article}
\title{Computer Aided Hyphenation for Italian and Modern Latin}
\author{Claudio Beccari}
\address{Dipartimento di Elettronica\\Politecnico di Torino\\ Turin, Italy}
\netaddress{beccari@polito.it}
%
% New environment "comment"
%
\newenvironment{comment}{\begingroup\setbox0\vbox\bgroup}{\egroup\endgroup}
%
% New environment to typeset on three columns
%
\newenvironment{trecolonne}{%                   Opening commands
\hbadness=10000 \vbadness=10000
\widowpenalty=0 \clubpenalty=0% Necessary to counteract ltugboat.sty settings
\dimen0=\textwidth \advance\dimen0 -2\columnsep
\divide\dimen0 by 3
\setbox0\vbox\bgroup\hsize\dimen0\parindent 1em
bbbb\par
}{%                                             Closing commands
\par\egroup
\setbox0=\vbox{\unvbox0\null}
\setbox6\vsplit0 to \baselineskip
\count255\ht0 \divide\count255 by \baselineskip \divide\count255 by 3
\dimen2=\count255\baselineskip \advance\dimen2\topskip
\global\setbox2\vsplit0 to\dimen2
\setbox2\vbox{\unvbox2}
\ifdim\ht2<\dimen2 \setbox2\vbox{\unvbox2\vsplit0 to \topskip}\fi
\global\setbox4\vsplit0 to\dimen2
\setbox4\vbox{\unvbox4}
\ifdim\ht4<\dimen2 \setbox4\vbox{\unvbox4\vsplit0 to \topskip}\fi
\setbox2\vtop{\unvbox2}
\setbox4\vtop{\unvbox4}
\setbox6\vtop{\unvbox0}
\noindent\box2 \hfill \box4 \hfill \box6}%      End of definition
%
\let \italiano\relax \let\latino\relax

\hbadness=5000
\begin{document}
\maketitle
\section*{Abstract}
After  an essential historical sketch of the evolution of latin into italian
and modern latin, the peculiarities of both languages are described so as to
understand the philosophy of the hyphenation patterns. The latter are one of
the few cases where the same set serves two different languages.

\section*{Sommario}  Dopo  aver delineato brevemente l'evoluzione del latino
verso l'italiano e il latino moderno, vengono descritte  le  caratteristiche
delle  due  lingue  in  modo  da  capire  la  filosofia dei {\it pattern} di
divisione in sillabe. Questi  {\it  pattern}  costituiscono  uno  dei  pochi
esem\-pi applicabile a due lingue differenti.

\section*{Summarium}  
Latini sermonis evolutione ad italianum et la\-ti\-num
modernum breviter exposita,  utrius  sermonis  spe\-cie\-ta\-tes  descriptae
sunt  ut  philosophia  de  {\it pattern} ad syllabas dividendi intelligatur.
Isti {\it pattern}  duobus  differentibus  sermonibus  applicabile  exemplum
sunt.

\section{Outline of historical evolution}
Classical  latin  as we study it in schools and universities is the language
that was used, especially in written form, by the authors of the  republican
period  and  of  the  very  beginning  of  the empire. Common people spoke a
similar language that was open to the contribution of new words  from  other
countries,  to  new  constructs  and  to  a  general  simplification  of the
inflection of nouns, adjectives and verbs.

Cicero  himself  was complaining about the fact that common people (the {\it
vulgus\/}) used to shorten the desinences leaving out the final  consonants,
and  used to palatalize the `c' and `g' followed by the front vowels `e' and
`i'. Those were the first signals of  the  autoctonous  evolution  of  latin
towards  the modern language; in the other parts of the Roman empire similar
evolutions were going on with a stronger influence of the  native  languages
over    which    latin   had  superimposed  itself;  the  invasions  of  the
``barbarians'' brought in peculiar  pronunciations  and  a  lot  of  lexical
additions.

Latin  decline was very slow because it was the scholar's, the chancellor's,
the notary public's language for many centuries, and it was and still is the
official  language  of the Roman Catholic Church; latin, in its modern form,
is the official language  of  the  Vatican  State,  and  the  daily  Vatican
newspaper,  {\it  L'Osservatore Romano,} is published mainly in italian, but
with frequent contributions in latin, even commercial adds! Modern latin  is
used  even  for  comics  books; I suggest Snoopy \cite{snoopy}, Mickey Mouse
\cite{MMouse}, Asterix  \cite{asterix}\footnote{The  former  two  books  are
intended  as  didactic  aids for teaching latin, and are fully accented with
both prosodic and rythmic marks.}.

Nowadays  latin  is  studied  in many countries as a regular subject both in
high school and in  universities;  in  Italy  it  is  not  classified  as  a
``foreign''  language  and  is  a  compulsory  subject both in classical and
scientific {\it licei} (high schools). In the  past,  latin  was  even  more
important  in the education of young people; forty years ago I started latin
in sixth grade and had eight years  of  it  through  junior  high  and  high
schools\footnote{I  frequented  the  {\it  liceo classico} and had also five
years of classical greek; now I have  an  engineering  degree  and  I  am  a
professor  of  electric circuit theory. I am very glad I had the opportunity
of completing my education by studying humanities for so long,  and  I  wish
the new generation could have the same.}.

From  the common people's language of the first century several regional and
local dialects evolved; in 960~A.D.\ there is the first document  explicitly
written  in  what  we  might already call italian \cite{migliorini}; several
documents, mostly including poems, were produced in the following centuries,
and by the end of the thirteenth century the masterpiece of Dante Alighieri,
the {\it Divina Commedia}, is  considered  the  main  landmark  of  the  new
language,  that  was already so mature as to be used in a poetic treatise of
history, philosophy and theology.

The  modernization  of  Dante's  language  took  place during the past seven
centuries, but compared  to  modern  italian  there  is  not  such  a  great
difference  as  between  the language used by Chaucer in his {\it Canterbury
Tales} and modern english; today's italian high  school  students  can  read
Dante's  poem  and  other even older texts with no more difficulty than that
required by any other conceptual text.

\section{Alphabet}
Italian  and  modern  latin  use the 26 letter alphabet that everybody knows
with the name of {\it latin alphabet\/}; actually there are some fine points
to consider with due attention.

\noindent  {\it  Italian.}  The  letters  J, K, X, Y, and W are used only in
technical terms and symbols, foreign names and some very specialized  words,
such  as  the international word {\it taxi}. J, K and Y survive in toponyms,
family names, and english style nick  names,  such  as  Stefy  for  Stefania
(Stephanie).  The  letter J (see also below) used to be employed in the past
as a graphic device to distinguish the semivowel role of the  letter  I,  so
that  you have {\it Ajmone} (family name) and you may write {\it Iugoslavia}
(modern spelling), {\it  Jugoslavia}  (old  fashioned  spelling),  and  {\it
Yugoslavia}  (international  spelling)  according  to  your  preference;  in
italian all three are correct and are pronounced exactly the same way.

Besides the above mentioned letters, there are five vowels, none of which is
mute: {\it a, e, i, o, u}, fifteen consonants: {\it b, c, d, f, g, l, m,  n,
p,  q,  r, s, t, v, z}, and one diacritical letter: {\it h}. The latter does
not correspond to any sound and is used only to mark half a dozen  words  in
order  to  distinguish them from similar ones that sound the same but have a
different meaning, to  mark  some  interjections,  and  to  mark  the  velar
pronunciation  of  `c'  and `g' when otherwise they would be palatalized. 

\begin{comment}
In  total  there  are  26  signs,  but,  in  spite  of  the modern times, in
elementary school they keep teaching that the italian alphabet  contains  21
letters;  in  facts  they  are having troubles with the children with exotic
names such as John, Katia, Xenobia, Yuri, Walter and the like; these  names,
due  to  the  influence of the mass media, are now very common (well\dots, I
know just one person named Xenobia, but there are other names containing the
letter `x').
\end{comment}

Except    for  a  dozen  among  articles,  prepositions  and  adverbs  (that
nevertheless are used quite often), all common words in italian end  with  a
vowel;  of  course  this  statement  does  not  apply  to  trade  marks, not
assimilated foreign words, technical terms, and the like.

Another  peculiarity  is that every consonant may occur in its doubled form,
and this corresponds to its  reinforcement  when  the  double  consonant  is
pronounced.  There  are  rare  instances of double vowels, but in this case,
contrary to what happens in english, they form different  syllables  instead
of  a  diphthong;  for  example,  {\it  zoologico}  can  be  divided in {\it
zo-o-lo-gi-co}.

\noindent  {\it Latin.} Classical latin missed J, U, and W, while V was used
throughout wherever now U or V are  used.  Since  the  very  beginning  this
anomaly  was passed by the scholars on into the spelling and printing of all
languages; capital V was used in all circumstances, while `v'  was  used  in
printing at the beginning of words and `u' in the middle or at the end. This
confusing habit was common to all western languages but fortunately  it  was
abandoned  starting  in  Holland  during  the sixteenth century; it lasted a
little more in Italy because of the wide use of latin,  but  was  eventually
done  away by the end of the seventeenth century. When Knuth \cite[reference
106]{knuth} cites Pacioli's {\it Diuine Proportione}, published in Venice in
1509,  he reports that title with the spelling of the original printing, but
the pronunciation at that time already implied the consonant  V  instead  of
the vowel~U.

In the middle ages and in the early times of printing there was the habit of
using `j' instead of `i' in those  cases  where  the  letter  `i'  formed  a
diphthong  with  the  following  vowel;  it  was  just  a  graphic  trick to
distinguish the two roles of the letter `i', and it was so  successful  that
it was adopted also in other languages; this is the reason why even today we
spell {\it junior} instead of {\it  iunior},  although  the  latter  is  the
formal latin spelling.

Modern  latin  uses  both U and V in the proper positions, while J and W are
used only in foreign names and toponyms.

There  are  six vowels: {\it a, e, i, o, u, y} and eighteen consonants: {\it
b, c, d, f, g, h, k, l, m, n, p, q, r, s, t, v, x, z}.  The  ligatures  {\it
\ae,  \oe}  do  not  belong  to latin; they were introduced in the sixteenth
century in France and in England, and after  that  they  enjoyed  a  certain
popularity  also  in  latin,  but  in  modern usage, as well as in classical
latin, these two diphthongs are spelled with separate letters.

\section{Accents}
{\it  Italian.} In italian accents are used very sparingly; it is compulsory
to mark with a suitable accent the last vowel of polysyllabic oxitone  words
(those  that receive the stress on the last syllable), and to mark some well
known and specified monosyllabic words that contain  a  diphthong.  This  is
standardized by the Regulation UNI~6015 \cite{6015}.

Contrary to spanish and portuguese, in italian there is no necessity to mark
proparoxitone words with an accent, although the best grammars recommend  to
do  so.  In  practice,  if  you  exclude  oxitone  words  (where accents are
compulsory) and paroxitone words (where accents are not required), the other
ones  {\it  may}  be marked with an accent only when a different position of
the stress might change the meaning; for example {\it l\`avati} means  `wash
yourself'  while {\it lav\`ati} is the masculine plural of `washed'; in this
circumstance it is advisable to mark the first case  unless the  meaning  of
the rest of the sentence does not make clear which case is implied. Although
the `Sommario' of this article contains five proparoxitone words, no accents
were used.

The  accent  can  be  used  also for denoting the open or closed nature of a
vowel (only for  tonic  `e'  and  `o'),  but  this  use  is  found  only  in
dictionaries and grammars; a good grammar will certainly point out that {\it
c\`olto} (picked up) is different from  {\it  c\'olto}  (educated),  but  in
practice  the  meaning  is  determined  by  the  context  while  the  actual
pronunciation very strongly depends on the regional origin of the speaker.

The  grave (\`{}) accent is used on any vowel, while the acute (\'{}) accent
may be used only on the vowel `e'  (and  on  the  vowel  `o',  but  only  in
optional  situations) when it has a closed sound. Most Italians are not even
aware of this choice; when they hand write, they just put any kind of  small
surd  on  the vowel to be accented, and by so doing they intend to mark only
the stress; the tonic value of the accent is used only in  dictionaries  and
grammars, while in printing the difference is maintained only for the letter
`e' in oxitone words more as a tribute to the tradition than for  an  actual
semantic  necessity. 

\begin{comment}
Some  fancy character sets have both accents merged into a single horizontal
bar. 
\end{comment}

When  the  accent  is  compulsory  and  upper  case letters are used, if the
character set does not contain accented vowels, it is  accepted  to  use  an
apostrophe:  UNITA' (unity) in place of UNIT\`A; this practice is considered
bad style in typesetting, but is used quite often in advertising.

The  diaeresis (\"{}) and the circumflex (\^{}) are not used anymore; in the
past the diaeresis was  used  in  poetry  to  split  a  diphthong,  and  the
circumflex   had  several  meanings  such  as,  for  example,  to  mark  the
contraction of two `i' into one sign in those  plurals  that  centuries  ago
were  spelled  with a double `i': {\it studii} (studies, two centuries ago),
{\it stud\^\i} (one century ago), {\it studi} (modern).


\noindent{\it Latin.} In latin no accents are used; the breve (\u{}) and the
long (\={}) accents are  used  only  in  dictionaries,  grammars  and  where
prosody  is  dealt  with. The diaeresis is sometimes used in grammars and in
prosody to mark the splitting  of  a  diphthong:  {\it  a\"er}  (air),  {\it
po\"eta} (poet).

\section{Apocope and aphaeresis}
{\it  Italian.}  In italian the dropping of one or more initial letters in a
word (aphaeresis)  takes  place  only  in  poetry  and  is  marked  with  an
apostrophe preceded by a white space.

The  loss  of one or more terminal letters in a word (apocope) either is not
marked at all (see in the `Sommario' {\it aver} in place of  {\it  avere\/})
or  it is marked with an apostrophe when it corresponds to a vocalic elision
(see above {\it l'evoluzione} in place of {\it la  evoluzione\/})  or  to  a
complete  syllabic  apocope.  The  latter  case  is  very unusual, while the
vocalic elision is very frequent, so that  this  case  must  be  taken  care
properly  when  dealing with hyphenation; the rules stated in the Regulation
UNI~6461 \cite{6461} require that when a line ends with an apostrophe,  this
{\it  must  not}  be replaced back with the vowel it originally replaced. In
the past, not too long ago, for example when I was in elementary school, the
opposite  rule  was in use, so that there are occasional discussions between
the old styled generation and the new one. Nevertheless even to  day  it  is
considered  bad  style  to  end a line with an apostrophe, and in typography
this practice is tolerated only when the line width is quite  small,  as  in
the daily newspapers narrow columns.

\noindent{\it  Latin.} I do not know of any case where apocope or aphaeresis
are marked in any visible way; actually I am  almost  sure  that  these  two
spelling behaviours are not legal in latin.

\section{Diphthongs}
{\it  Italian.}  In  italian  a diphthong is formed by any vowel preceded or
followed by an {\it unstressed} closed vowel (`i' or `u'); so we have:
 \begin{center}
\it  ia, ie, io,  ai, ei, oi  \\
     ua, ue, uo,  au, eu, ou  \\
             iu,  ui
\end{center}

Italian  diphthongs  are  always  pronounced  maintaining  the sounds of the
individual vowels, and the closed vowel plays the role of a semivowel  or  a
glide.

There  are  also  groups  of  three  vowels that contain two semivowels or a
semivowel and a glide:
 \begin{center}\it
iuo, uie \\
ieu, uoi, iei
\end{center}

An  `i'  (possibly also an `u', but I can't find examples) surrounded by two
open vowels behaves always as  a  semivowel,  so  it  always  starts  a  new
syllable.

\noindent{\it Latin.} In latin there are more or less the same diphthongs as
in italian with the addition of
\begin{center}\it  
ae,  oe  
\end{center} 
that  one or two centuries ago were written with the corresponding ligatures
{\it \ae, \oe}; in modern latin the pronunciation of both  these  diphthongs
is  given  by  a  single  open `e'\footnote{I have seen a reproduction of an
italian book printed in Venice in  the  sixteen  century  where  both  these
diphthongs  where  replaced  by  their  sound  given  by  the  letter `e'.}.
Furthermore in some words of greek origin, latin may have the diphthong {\it
yi}, for example {\it Harpyia} \cite{manna}\footnote{One might think that it
would be the same to consider the vowel `y' and the  diphthong  `ia',  since
the  pronunciation would be practically the same; but if you look at it from
the prosody point of view, the  situation  becomes  completely  reversed;  a
diphthong  is  always  long  while  `y'  is always short, so that in prosody
Har-pyi-a becomes \={}\={}\u{}, while Har-py-ia becomes \={}\u{}\={}.}.

The main difference between italian and latin common diphthongs is that {\it
ia, ie, io, iu} behave as such in latin only when they are at the  beginning
of  a word or are preceded by another vowel; in any other case they are part
of two different syllables; in italian they are always diphthongs provided 
the `i' is unstressed.

\section{Di- and trigraphs}
{\it  Italian.}  In  italian  there  are groups of two or three letters that
imply a sound that is  not  implied  by  any  other  single  letter  of  the
alphabet; besides `c' and `g' modified with the diacritical `h', and `c' and
`g' modified with a diacritical `i'\footnote{In this  case  the  letter  `i'
does  not  form  a  diphthong  with  the following vowel but is used just to
palatalize the two consonants; under the  hyphenation  point  of  view  this
subtle difference may be ignored.} there are
 \begin{center}\it gn, gli, sc \end{center}
where  {\it  gn}  is pronounced as in french, or as the spanish {\it \~n} or
the portuguese {\it nh\/}; {\it sc} is pronounced as the  english  {\it  sh}
when is followed by a front vowel `e' or `i', and {\it gli} is pronounced as
the portuguese {\it lh} when it is not preceded by `n' and  is  followed  by
another  vowel  or  it is at the end of a word. These digraphs and trigraphs
must not be split by the hyphenation process.

\noindent{\it  Latin.}  In latin by itself there are no indivisible digraphs
or trigraphs, but since the classical times  the  transliteration  of  greek
words  required  {\it  th} in place of $\theta$, {\it rh} in place of $\rho$
(but {\it rrh} in place of $\rho\rho$), {\it ph} in  place  of  $\phi$,  and
{\it  ch}  in  place of $\chi$; therefore these digraphs can not be split by
the hyphenation process.

\section{Hyphenation}
{\it Italian.} The italian hyphenation rules are stated very simply as follows:
\begin{enumerate}

\item  every syllable contains at least one vowel\footnote{This rule applies
to all languages,  although  in  every  language  the  notion  of  vowel  is
different;  for  example  in  several  slavic  languages `r' is considered a
vowel. If  \TeX\  contained  a  provision  for  this,  the  bad  line  break
(compara-nds)  that  occurred  in  \TUB, vol.12, n.2, June 1991 at page 239,
first column, 6-7 lines from bottom, would not have taken place.}

\item diphthongs and `triphthongs' behave as one vowel

\item  a  consonant  followed by a vowel belongs to the same syllable as the
vowel

\item  one  or more consonants not followed by a vowel (at the end o a word,
possibly because of an apocope, or in technical terms, trade marks  and  the
like) belong to the same syllable as the preceding vowel

\item  when  a group of consonants is found, the hyphen position is the {\it
leftmost} one (even at the left of the whole group) such that the consonants
that remain on the right of the hyphen can be found also at the beginning of
an italian word;\label{cons}

\item  prefixes  and  suffixes  can  be ignored and the compound word may be
divided as if it were a single word; in any case the division  according  to
the  etymology is accepted; in practice this happens only with the technical
prefixes {\it dis-, post-, sub-, trans-,} which are not very common.

\end{enumerate}

Once  it  is  clear  what  is  a  consonant,  a  vowel,  a  diphthong  and a
`triphthong',  the  only  difficult  rule   to    apply    is    the    rule
number~\ref{cons};  but  with the help of a school dictionary one can always
find if there exists an italian word  starting  whith  a  certain  group  of
consonants.

The  point  is  that if you use a dictionary of too high a quality, you will
find words starting with almost  any  possible  group  of  consonants:  {\it
bdelio\footnote{Due to the extremely specialized nature of these words, I do
not give the translation in english, because  I  did  not  find  a  suitable
italian-english dictionary that reported them; I believe, though, that their
scholarly nature is such that with minor modifications they  exist  also  in
english  and  many  other languages.}, cnidio, ctenidio, ftalato, gmelinite,
pneumatico, psicosi, pteridina, tmesi}. But many of these words,  mostly  of
greek  origin,  do  not find their way into school dictionaries (except {\it
pneumatico} and {\it psicosi\/}), so that a  diligent  person  will  not  be
misled by too many technicalities and will find the correct division.

The  Italian Standards Institute, in order to avoid confusion in this matter
established the Regulation UNI~6461 \cite{6461}  that  lists  the  group  of
consonants  that  must  be  divided, table~\ref{t:6461}. This table does not
list the normal consonant divisions, that is
 \begin{itemize}

\item  digraphs  and  trigraphs  can {\it never} be divided, except {\it gn}
when it appears in a foreign word or in a word that derives from  a  foreign
one  and  where  the  two  letters are pronounced individually, such as {\it
Wagner, wagneriano,\dots}

\item geminated (double) consonants and {\it cq} must {\it always} be split

\item  a  liquid  (`l',  `r') or a nasal (`m',`n') is {\it always} separated
from a following consonant except for the cases shown in table~\ref{t:6461}

\item  any  consonant  is  {\it  never}  separated from the following liquid
except for the cases shown in table~\ref{t:6461}

\item  the  letter `s' is {\it never} separated from any following consonant
(unless it is another `s')

\end{itemize}

\begin{table}{\centering\tt
\begin{tabular}{|*5{c|}}\hline
b-d  & b-n  & b-s  & c-m  & c-n  \\
c-s  & c-t  & c-z  & d-g  & d-m  \\
d-v  & f-t  & g-m  & p-n  & p-s  \\
p-t  & p-z  & t-m  & t-n  & z-t  \\
g-fr & ld-m & ld-sp& l-st & mb-d \\
mp-s & nc-n & ng-st& n-scr& n-st \\
n-str& r-st & r-str& st-m &      \\
\hline
\end{tabular}\par}
\caption{Groups    of   consonants  that  can  be  split  across  syllables}
\label{t:6461} 
 \end{table}

The  Regulation  UNI~6461 states also the rules for the apostrophe, i.e.\ it
behaves as the vowel it replaces; line breaking (without hyphen) is  allowed
after  it when the line is very short, but it is bad style to do it, so that
line breaking is eliminated if  no  interword  space  is  left  between  the
apostrophe  and  the  following  word.


Italian  hyphenation  for  \TeX\  was  already  explained by D\'esarm\'enien
\cite{desarmenien}, but, although I wish I knew french as well as  he  knows
italian,  the  88  patterns  that  he created for italian were good only for
consonants while completely  ignored  diphthongs  and  `triphthongs';  in  a
previous  version  I  prepared,  150 patterns were needed to perform italian
hyphenation correctly.

For  the  rest the regulation is already made in such a way as to synthesize
the hyphenation patterns \TeX\ requires, without the need  of  running  {\tt
patgen};  of  course  some  care must be exercised in order to avoid strange
situations and in order to replace \TeX\  inability  to  distinguish  vowels
from consonants.

With    the    advent  of  Version  3.xx  of  \TeX\  it  is  better  to  set
\verb"\righthyphenmin" to the value 2, because there is no need  to  protect
the hyphenation algorithm from the mute vowels (`e') that are so frequent in
english; of course it is not good style to go on a new line  with  just  two
letters,  but  this  is  so  rare  that it is much better to give \TeX\ more
chances to  find  suitable  line  break  points  than  to  protect  it  from
situations that in italian never take place.

Another reason for choosing this reduced value for \verb"\righthyphenmin" is
due to the accents; it was pointed out that in practice italian has accents,
if any, only on the last ending vowel of a word. It is known that \TeX\ does
not hyphenate a word after an accent control sequence, but this drawback has
a  negligible  influence  on italian since after the accent control sequence
the word may have just one letter; when accented letters will find their way
into the 256 symbol character sets, this simple drawback will be eliminated,
but even with the actual limitations (unless virtual fonts are  used)  \TeX\
peculiarity  is  of no influence; I admit that {\it virt\`u} (virtue) cannot
be hyphenated  because  is  too  short  (it  could  be  hyphenated  as  {\it
vir-t\`u\/}),  while  there  are  no problems with longer words, for example
{\it qualit\`a} (quality) is hyphenated by \TeX\ as  {\it  qua-lit\`a},  the
full  possibility  being  {\it  qua-li-t\`a}. But \TeX\ gives correctly {\it
per-ch\'e} (because), {\it af-fin-ch\'e} (so that), and so on.

There are no known problems with the synthesized patterns listed at the end;
the only point that leaves me partially  unsatisfied  but  is  grammatically
perfectly  correct,  is  the fact that technical prefixes such as {\it dis-,
post-, sub-, trans-} must be explicitly  separated  with  \verb"\-"  if  one
wants to stress their specific prefix nature. See below the solution for the
same problem in latin.



\noindent{\it  Latin.}  The  patterns  that  are listed at the end include a
subset that was originally designed just for italian; with a little  thought
and  few additions the pattern set was extended so as to include also modern
latin.

For  what  concerns  diphthongs,  italian  and  latin diphthongs were merged
together under the assumption that \TeX\  is  not  supposed  to  find  every
possible  break point but only legal break points, so that if two vowels are
treated as a diphthong even if they belong to two different  syllables,  the
only  drawback  is that you miss a legal break point but you do not make any
wrong division. More over most Italian readers  feel  uncomfortable  when  a
break  point  is  taken  such that the new line starts with a vowel (this is
certainly not the case with anglophone readers) so that the extension of the
set  of  diphthongs  of    either  language  does not bother neither italian
readers, nor latin ones. The declaration of  {\it  \ae}  and  {\it  \oe}  as
letters with their \verb"\lccode" allows the hyphenation of words containing
such ligatures, although their use is discouraged.

For what concerns consonant groups there is no regulation as for italian; my
grammar \cite{manna} claims that latin hyphenation is  done  as  in  italian
(except  for  what  concerns  prefixes  and  suffixes  that  must be divided
etymologically) but in latin there are  consonant  groups  that  in  italian
never occur.

In  order  to  find  out how unusual consonant groups are treated in latin I
examined  an  important  scholar's  book  \cite{merk},  the  bilingual   New
Testament in greek and latin ``apparato critico instructum'', reprinted as a
``reeditio  photomechanica  ex  typographia~\dots,  Romae''  and  for  which
``omnia  iura  reservantur'';  clearly  this  is  modern latin, although the
book's contents, the latin part, contains  the  well  known  text  that  was
translated  from  greek  and  aramaic  by  several  authors  across  several
centuries and  copied  by  different  copyists  in  many  codices  that  are
preserved  all  over the world. This critical edition is intended as a study
material and is particularly cured in the language and the spelling for  the
very purpose of the book.

By  examining  the  hyphenations  of  this  book  I  could  list a series of
consonant groups, and I could realize that the digraph {\it  gn}  (which  is
such  in  italian but it is not supposed to be one in latin) was treated not
uniformly so as to have both {\it reg-num} and {\it re-gnum}. I  decided  to
chose  the  second  form  of  hyphenation  for  two  reasons: a) it does not
conflict with the italian rule, and b) the pronunciation recommended to  the
clergy  and  that  is  being  used in the catholic universities, seminaries,
monasteries, etc., corresponds to the italian one.

Also  the letter `s' is not treated uniformly; it is generally treated as in
italian, but there are cases where it is treated as in english; for  example
{\it blasphemia} (blasphemy) is hyphenated as {\it blas-phe-mia}. Since this
does not conflict with the italian rule (in this language the group `sph' is
missing)  a  suitable  pattern  was  generated  in  order  to cope with such
situations.


Some attention was given to the prefixes and suffixes in order to find a way
to separate them correctly according to their etymology; for  what  concerns
prefixes,  these  must be separated regardless of the groups of letters that
get split away, provided that the prefix did not loose its  final  vowel  by
elision  with  the  initial  vowel  of the compound word second element. For
example the prefix {\it  paene-}  (almost)  looses  the  last  `e'  in  {\it
paeninsula}  and therefore the whole word is treated as a single word and is
hyphenated {\it pae-nin-su-la}.

It was possible to find suitable patterns for certain instances of {\it ab-,
ad-, ob-, trans-}, for the prefixes {\it abs-, dis-, circum-, sub-}, and for
the  suffixes {\it -dem, -que} but the problem remains, although it shows up
not so often.

The  solution  can  be  found  in  a  macro  (already described by J.~Braams
\cite{braams}) that has been in use by the German \TeX\ users, which have to
cope  all  the  time  with  compound words that need a little help for their
correct hyphenation:
 \begin{verbatim}
\def\allowhyphens{\penalty\@M\hskip\z@}
\def~#1{\if\string#1-
           \allowhyphens\-\allowhyphens
        \else
           \penalty\@M\ #1%
        \fi
}
\end{verbatim}

Here  this  macro  appears  in  a  modified  form; in the german version the
character \verb|"| (instead of \verb|~|) was made active  and  was  given  a
complex  definition so as to treat the umlaut in the proper way and to cover
several other situations that occur in german. This implies several  changes
to  be  made  here and there in the definitions of \plain, in particular the
double quote must be added to the list of special characters so as  to  deal
with them in a consistent way when typesetting in verbatim mode. I preferred
to give a new definition to the tie  character  \verb|~|,  that  is  already
listed  among the special characters; this new definition performs the usual
tie function except when is followed by the hyphen character; in the  latter
case  the  sequence \verb|~-| inserts a special discretionary break that has
the property that normal hyphenation takes place in the rest  of  the  word;
remember,   in  facts,  that  the  standard  sequence  \verb|\-|  inserts  a
discretionary break but inhibits hyphenation in the rest of the word.

Therefore,  if  wrong prefix or suffix hyphenations are found in the drafts,
it is possible to correct (or to write it  that  way  since  the  beginning)
\verb|con~-iungo,  ob~-iurgo|  so  that  the possible hyphenation points are
{\it con-iun-go, ob-iur-go}.


\begin{figure*}
\begin{trecolonne}\italiano
La lingua italiana e le lingue cosiddette romanze o neolatine, cio\`e lingue
derivate anch'esse dal latino (francese,  spagnolo,  portoghese,  rumeno  ed
altre minori), si fanno risalire all'idioma, che al tempo dell'impero romano
era  parlato  nella  penisola  italiana,  nelle  regioni  del   Mediterraneo
occidentale e nella Dacia, l'odierna Romania.

Tracce  evidentissime  si  osservano  ancor  oggi non soltanto nel lessico e
nella morfologia del gruppo linguistico neolatino, ma anche in altre  lingue
europee,  quelle  del  gruppo  anglo-sassone, come conseguenza dell'influsso
diretto o indiretto esercitato dalla lingua di Roma sugli idiomi particolari
dei popoli nordici.

Per  quel  che  riguarda la lingua italiana, essa si collega direttamente al
{\it sermo vulgaris la\-ti\-nus,} cio\`e al latino parlato comunemente dalle
famiglie e in pubblico nei quotidiani rapporti di commercio e di affari.
 \end{trecolonne}
\caption[]{Example of italian text typeset in narrow columns (from
\cite{manna})}
 \medskip
\begin{trecolonne}\latino
Et  sicut Moyses exaltavit serpentem in deserto, ita exaltari oportet Filium
hominis, ut omnis, qui  credit  in  ipsum,  non  pereat,  sed  habeat  vitam
aeternam.  Sic enim Deus dilexit mundum, ut Filium suum unigenitum daret, ut
omnis qui credit in eum non pereat, sed  habeat  vitam  aeternam.  Non  enim
misit  Deus Filium suum in mundum, ut iudicet mundum, sed ut salvetur mundus
per ipsum. Qui credit in eum, non  iudicatur;  qui  autem  non  credit,  iam
iudicatus  est, quia non credit in nomine unigeniti Filii Dei. Hoc est autem
iudicium, quia lux venit in mundum, et  dilexerunt  homines  magis  tenebras
quam  lucem;  erant  enim  eorum mala opera. Omnis enim, qui male agit, odit
lucem et non venit ad lucem, ut non arguantur opera eius;  qui  autem  facit
veritatem,  venit  ad  lucem,  ut manifestentur opera eius, quia in Deo sunt
facta.
 \end{trecolonne}
\caption[]{Example of latin text typeset in narrow columns (J\,3,14-21)}
\end{figure*}


\section{Generation of the format file}
In  the appendix the file {\tt italat.tex} is listed and the patterns may be
checked against the rules that have been stated in  the  previous  sections.
Special attention was given to the groups {\it ps} and {\it pn}, because the
table~\ref{t:6461} states that they must  be  separated,  but  the  compound
words  with  {\it  psic-}  (example {\it parapsicologia\/}) and {\it pneum-}
(example {\it pseudopneumococco\/}) must not be hyphenated after the `p'.

The  ligatures  `\ae'  and  `\oe'  have  been  included  with  the \verb|^^|
notation, because the patterns can not contain control sequences; this poses
no  problems to the final user, because the hyphenation algorithm is applied
after all macro expansions have been reduced to non expandable tokens.

The pattern list is preceded by some definitions:
\begin{itemize}

\item  the  category,  lower  case  and  upper case code definitions for the
ligatures `\ae' and `\oe' so that they can be used in latin text;  I  stress
again  that these ligatures should not be used, except when quoting verbatim
some text where they have been used.

\item the definition of the special control sequence \verb|~-|;

\item  the  definition  of  the  new  language  ``italian'' with the command
(\verb|\italiano|)  that  invokes  all  the  auxiliary   definitions;    the
apostrophe  character  must be given its \verb"\lccode=39" so as to treat it
as a normal letter and as the vowel it replaces.


\item the command for latin (\verb'\latino', ablative and short for ``latino
sermone'') is simply \verb'\let' to be identical with \verb'\italiano'.

\end{itemize}

The  patterns  are enclosed within a group so that the \verb'\lccode' of the
apostrophe and the codes for the ligatures `\ae' and `\oe' remain local  and
do  not  mix  things up with the default language and/or with the previously
defined languages.

Adding  these  hyphenation  patterns  to  the  format  that  has one or more
languages already defined is not a heavy overhead; if you  add  italian  and
latin  to  the default language `english' you do not need a large version of
\TeX; the statistics, after running {\tt initex}, say that  the  hyphenation
trie  is of size 6336 with 220 ops, 181 of which are used for english and 39
for italian and latin; italo-latin hyphenation requires  just  202  patterns
(some  of which probably never occur in practice) against the 4447 needed in
english.

\section{Conclusion}
The  hyphenation  patterns  valid  for  both  italian  and  latin  have been
generated directly from the grammar hyphenation  rules;  for  what  concerns
italian  the  set  of  patterns  (a  subset  of  that shown in the file {\tt
italat.tex} reported in the appendix) has been in use for two years  in  the
Institution  where  I  work, and after a short period of careful observation
and debugging it performed absolutely without errors of any  kind.  Although
the  italian rules allow to hyphenate a compound word as if it were a simple
one, some prefixes that are mainly used in technical terms may be explicitly
hyphenated   with  the  help  of  the  special  discretionary  hyphen  macro
\verb|~-|.

For  what  concerns latin the there is less experience but the impression is
that also in this language there are no  hyphenation  errors;  any  how  the
author  is grateful to anyone that might report suggestions and corrections.
The special discretionary hyphen macro \verb|~-| is very useful for prefixes
and  suffixes  and  must  be  used  whenever  unusual consonant clusters are
generated by the apposition of a prefix or a suffix.


In  Figures~1  and~2  two  examples  show the performance of the hyphenation
algorithm in italian and in latin when the line width  is  very  small;  the
line  breaking  tolerance is the default one (200) and in each example there
are a just couple of underfull hboxes.

I  am  pleased to express my thanks to the Nuns of the Benedictine Monastery
of Viboldone (S.~Giuliano, Milano, Italy) who helped me very much with their
experience in typesetting latin and other ancient languages.




\appendix
\onecolumn
\section{The {\tt italat.tex} file}
This  file must be input after the last line of the file {\tt plain.tex} (or
{\tt lplain.tex} for \LaTeX); the  definitions  given  before  the  list  of
patterns  are  better  located in the format file, so they are valid for any
style and there is no possibility to forget them out.
 \small
\begin{verbatim}
%%%%%%%%%%%%%%%%%%%%%%%%%%%%%%%%%%%%%%%%%%%%%%%%%%%%%%%%%%%%%%%%%%%%%%%%%%%%
%
%                        F I L E    I T A L A T . T E X
%
%                  Hyphenation patterns for Italian and Latin
%%%%%%%%%%%%%%%%%%%%%%%%%%%%%%%%%%%%%%%%%%%%%%%%%%%%%%%%%%%%%%%%%%%%%%%%%%%%
%          Prepared by Claudio Beccari, Politecnico di Torino, Italy
%                           e-mail beccari@polito.it
%
% Version date  27 august 1991
%
% Useful definitions 
%
\def\catcodeAE{\catcode 26=11 \catcode 29=11 \lccode 29=26    % Ligature ae,AE
               \uccode 29=29  \lccode 26=26  \uccode 26=29
               \catcode 27=11 \catcode 30=11 \lccode 30=27    % Ligature oe,OE
               \uccode 30=30  \lccode 27=27  \uccode 27=30}
\makeatletter %                  Because when this file gets read @ is "other"
\def\allowhyphens{\penalty\@M\hskip\z@}
\gdef~#1{\if\string#1-\allowhyphens\-\allowhyphens
           \else \penalty\@M\ #1\fi}
\makeatother%                                             Restore @ to "other"
%
% A number is given to italian/latin hyphenation
%
\newlanguage\italian
%
% The commands \italiano and \latino are defined
%
\def\italiano{\language=\italian \righthyphenmin=2 \lccode`\'=39 \catcodeAE}
\let\latino\italiano
%
% The patterns are defined within a group so that the \lccode of the apostrophe
% remains local and does not interfere with other languages
%
{\language\italian \catcodeAE \lccode`\'=39
%
\patterns{
.a2b2s3  .a2b3l
.o2b3l   .o2b3m .o2b3r      .o2b3s
.an1ti3  .a2p3n .di2s3ci3ne .cir1cu2m3 .wa2g3n
.ca4p5s  .pre3i .pro3i
.ri3a    .ri3e  .re3i       .ri3o      .ri3u
.su4b3lu .su4b3r 2s3que.    2s3dem.
3p4si3c4 3p4neu1
^^Z1     ^^[1                                          %   Ligatures ae and oe
a1a   a2e    a2i    a2j    a1o   a2u  a2y              %   Diphthongs
a2y3o a3i2a  a3i2e  a3i2o  a3i2u ae3u
e1a   e1e    e2i    e2j    e1o   e2u  e2y e3iu
i2a   i2e    i1i    i2o    i2u   io3i
o1a   o2e    o2i    o2j    o1o   o2u  o2y
o3i2a o3i2e  o3i2o  o3i2u
u2a   u2e    u2i    u2o    u1u   uo3u
1b2   2b3b   4b3d   2b3n   2b3t                        %   Consonant groups
      2b3s4a 2b3s4e 2b3s4i 2b3s4o 2b3s4u  2b3s4t   u2b3s4c
1c2   2c3c   2c3m   2c3n   2c3q  2c3s  2c3t  2c3z  2ch3h
1d2   2d3d   2d3g   2d3m   2d3s  2d3v  4d3w
1f2   2f3f   2f3t
1g2   2g3g   2g3d   2g3f   2g3m  2g3s
1h2   1j2    2j3j   1k2    2k3k
1l2a  1l2e   1l2i   1l2j   1l2o  1l2u
      1l2l3l l3f4t  1l'    2l4l3m      1l2^^Z 1l2^^[
1m2   2m3m   2m3b   2m3p   2m3l  2m3n  2m3r   2m4p3s 2m4p3t 4m3w
1n2a  1n2e   1n2i   1n2j   1n2o  1n2u  2n3n   n2c1n  2n1l
      n2g3n  2n1r   n2s3m  n2s3f 2n'   1n2^^Z 1n2^^[
1p2   2p3p   2p3s   2p3n   2p3t  2p3z  2ph3p  2ph3t  2s3p2h
1q2   2q3q
1r2a  1r2e   1r2i   1r2j   1r2o  1r2u  1r2h   1r2^^Z 1r2^^[
1s2   2s3s   2st3m
1t2   2t3t   4t3m   2t3n   1t'   4t3w
1v2   2v3v   1w2    2w3w   wa4r
1x2a  1x2e   1x2i   1x2o   1x2u  2x3x  1x2^^Z  1x2^^[
y2a   y2e    y2i    y2o    y2u
1z2   2z3z   2z3t   1z'    }}
\end{verbatim}


\normalsize
\begin{thebibliography}{99}

\bibitem{snoopy}  Schulz  C.M., {\it Insuperabilis Snupius}, translated into
latin by G.~Angelino, European Language Institute, Recanati, Italy, 1984

\bibitem{MMouse}  Walt  Disney,  {\it  Michael  Musculus et Regina Africae},
translated into latin by C.~Egger, European  Language  Institute,  Recanati,
Italy, 1986

\bibitem{asterix}  Goshinny  and Uderzo, {\it Asterix gladiator}, translated
into latin  by  K.H.G.~von  Rothenburg  ({\it  Rubricastellanus\/}),  Delta,
Stuttgart, 1978

\bibitem{migliorini}  Migliorini  B.M.,  {\it Storia della lingua italiana},
(History of the italian language), Sansoni, Firenze 1963

\bibitem{knuth}  Graham  R.L.,  Knuth  D.E.,  Patashnik  O.,  {\it  Concrete
mathematics}, Addison-Wesley Publ. Co., Reading, Mass., 1989 (3rd printing)

\bibitem{6461}    {\it  Divisione  delle  parole  in  fin  di  linea}  (Word
hyphenation at the end of a line), published by UNI, Ente Nazionale Italiano
di Unificazione, Milano, 1969

\bibitem{6015}  {\it  Segnaccento  obbligatorio nell'ortografia della lingua
italiana} (Obbligatory accent marks for the correct spelling of the  italian
language),  bublished  by  UNI,  Ente  Nazionale  Italiano  di Unificazione,
Milano, 1967

\bibitem{desarmenien}  D\'esarm\'enien  J.,  ``The  use  of \TeX\ in French:
hyphenation and typography'' in {\it \TeX\  for  scientific  documentation},
D.~Lucarella ed., Addison-Wesley Publ.\ Co., Reading, Mass., 1985

\bibitem{manna}  Manna  F., {\it Il latino ieri e oggi} (Latin yesterday and
today), Signorelli, Milano, 1985

\bibitem{merk}  {\it  Novum Testamentum Graece et Latine} (The New Testament
In Greek and Latin),  A.~Merk S.J.\ ed., Istituto Biblico Pontificio,  Roma,
1984

\bibitem{braams}  Braams  J.,{\it  Babel, a multilingual style option system
for  use  with  \LaTeX's  standard  document  styles},  \TUB\  vol.12,  n.2,
June~1991, pp.~291-301

\end{thebibliography}

\makesignature
\end{document}

