% title: Hyphenation for Tamil
% copyright: Copyright (C) 2016 Santhosh Thottingal (santhosh dot thottingal at gmail dot com)
% notice: This file is part of the hyph-utf8 project.
%     See http://www.hyphenation.org for more project.
%     Original source https://github.com/santhoshtr/hyphenation/
% language:
%     name: Tamil
%     tag: ta
% licence:
%     - This file is available under any of the following licences:
%     -
%         name: MIT
%         url: https://opensource.org/licenses/MIT
%         text: >
%             Permission is hereby granted, free of charge, to any person
%             obtaining a copy of this software and associated documentation
%             files (the "Software"), to deal in the Software without
%             restriction, including without limitation the rights to use,
%             copy, modify, merge, publish, distribute, sublicense, and/or sell
%             copies of the Software, and to permit persons to whom the
%             Software is furnished to do so, subject to the following
%             conditions:
%
%             The above copyright notice and this permission notice shall be
%             included in all copies or substantial portions of the Software.
%
%             THE SOFTWARE IS PROVIDED "AS IS", WITHOUT WARRANTY OF ANY KIND,
%             EXPRESS OR IMPLIED, INCLUDING BUT NOT LIMITED TO THE WARRANTIES
%             OF MERCHANTABILITY, FITNESS FOR A PARTICULAR PURPOSE AND
%             NONINFRINGEMENT. IN NO EVENT SHALL THE AUTHORS OR COPYRIGHT
%             HOLDERS BE LIABLE FOR ANY CLAIM, DAMAGES OR OTHER LIABILITY,
%             WHETHER IN AN ACTION OF CONTRACT, TORT OR OTHERWISE, ARISING
%             FROM, OUT OF OR IN CONNECTION WITH THE SOFTWARE OR THE USE OR
%             OTHER DEALINGS IN THE SOFTWARE.
%     -
%         name: LGPL
%         version: 3
%         or_later: true
%         url: http://www.gnu.org/licenses/lgpl.html
%     -
%         name: GPL
%         version: 3
%         or_later: true
%         url: http://www.gnu.org/licenses/gpl.html
%
\patterns{
% GENERAL RULE
% Do not break either side of ZERO-WIDTH JOINER  (U+200D)
2‍2
% Break on both sides of ZERO-WIDTH NON JOINER  (U+200C)
1‌1
% Break before or after any independent vowel.
1அ1
1ஆ1
1இ1
1ஈ1
1உ1
1ஊ1
1எ1
1ஏ1
1ஐ1
1ஒ1
1ஓ1
1ஔ1
% Break after any dependent vowel, but not before.
ா1
ி1
ீ1
ு1
ூ1
ெ1
ே1
ை1
ொ1
ோ1
ௌ1
% Break before or after any consonant.
1க
1ங
1ச
1ஜ
1ஞ
1ட
1ண
1த
1ந
1ப
1ம
1ய
1ர
1ற
1ல
1ள
1ழ
1வ
1ஷ
1ஸ
1ஹ
% Do not break before any consonant + virama.
2க்1
2ங்1
2ச்1
2ஞ்1
2ட்1
2ண்1
2த்1
2ன்1
2ந்1
2ப்1
2ம்1
2ய்1
2ர்1
2ற்1
2ல்1
2ள்1
2ழ்1
2வ்1
2ஷ்1
2ஸ்1
2ஹ்1
% Do not break before anusvara, visarga and length mark.
2ஂ1
2ஃ1
2ௗ1
% Do not break before virama but break after virama.
2்1
}
