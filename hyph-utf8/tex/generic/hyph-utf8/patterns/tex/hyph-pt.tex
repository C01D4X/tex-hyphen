% This file is part of hyph-utf8 package and resulted from
% semi-manual conversions of hyphenation patterns into UTF-8 in June 2008.
%
% Source: pthyph.tex (Version 1.2, 1996-07-21 - date in file)
% Author: Pedro J. de Rezende <rezende at dcc.unicamp.br>, J.Joao Dias Almeida <jj at di.uminho.pt>
%
% The above mentioned file should become obsolete,
% and the author of the original file should preferably modify this file instead.
%
% Modifications were needed in order to support native UTF-8 engines,
% but functionality (hopefully) didn't change in any way, at least not intentionally.
% This file is no longer stand-alone; at least for 8-bit engines
% you probably want to use loadhyph-foo.tex (which will load this file) instead.
%
% Modifications were done by Jonathan Kew, Mojca Miklavec & Arthur Reutenauer
% with help & support from:
% - Karl Berry, who gave us free hands and all resources
% - Taco Hoekwater, with useful macros
% - Hans Hagen, who did the unicodification of patterns already long before
%               and helped with testing, suggestions and bug reports
% - Norbert Preining, who tested & integrated patterns into TeX Live
%
% However, the "copyright/copyleft" owner of patterns remains the original author.
%
% The copyright statement of this file is thus:
%
% BSD 3-Clause License (https://opensource.org/licenses/BSD-3-Clause):
% 
% Copyright (c) 1987, Pedro J. de Rezende (rezende@ic.unicamp.br) and J.Joao Dias Almeida (jj@di.uminho.pt)
% 
% All rights reserved.
% 
% Redistribution and use in source and binary forms, with or without
% modification, are permitted provided that the following conditions are met:
%     * Redistributions of source code must retain the above copyright
%       notice, this list of conditions and the following disclaimer.
%     * Redistributions in binary form must reproduce the above copyright
%       notice, this list of conditions and the following disclaimer in the
%       documentation and/or other materials provided with the distribution.
%     * Neither the name of the University of Campinas, of the University of
%       Minho nor the names of its contributors may be used to endorse or
%       promote products derived from this software without specific prior
%       written permission.
% 
% THIS SOFTWARE IS PROVIDED BY THE COPYRIGHT HOLDERS AND CONTRIBUTORS "AS IS" AND
% ANY EXPRESS OR IMPLIED WARRANTIES, INCLUDING, BUT NOT LIMITED TO, THE IMPLIED
% WARRANTIES OF MERCHANTABILITY AND FITNESS FOR A PARTICULAR PURPOSE ARE
% DISCLAIMED. IN NO EVENT SHALL PEDRO J. DE REZENDE OR J.JOAO DIAS ALMEIDA BE
% LIABLE FOR ANY DIRECT, INDIRECT, INCIDENTAL, SPECIAL, EXEMPLARY, OR
% CONSEQUENTIAL DAMAGES (INCLUDING, BUT NOT LIMITED TO, PROCUREMENT OF SUBSTITUTE
% GOODS OR SERVICES; LOSS OF USE, DATA, OR PROFITS; OR BUSINESS INTERRUPTION)
% HOWEVER CAUSED AND ON ANY THEORY OF LIABILITY, WHETHER IN CONTRACT, STRICT
% LIABILITY, OR TORT (INCLUDING NEGLIGENCE OR OTHERWISE) ARISING IN ANY WAY OUT
% OF THE USE OF THIS SOFTWARE, EVEN IF ADVISED OF THE POSSIBILITY OF SUCH DAMAGE.
%
% If you want to change this file, rather than uploading directly to CTAN,
% we would be grateful if you could send it to us (http://tug.org/tex-hyphen)
% or ask for credentials for SVN repository and commit it yourself;
% we will then upload the whole "package" to CTAN.
%
% Before a new "pattern-revolution" starts,
% please try to follow some guidelines if possible:
%
% - \lccode is *forbidden*, and I really mean it
% - all the patterns should be in UTF-8
% - the only "allowed" TeX commands in this file are: \patterns, \hyphenation,
%   and if you really cannot do without, also \input and \message
% - in particular, please no \catcode or \lccode changes,
%   they belong to loadhyph-foo.tex,
%   and no \lefthyphenmin and \righthyphenmin,
%   they have no influence here and belong elsewhere
% - \begingroup and/or \endinput is not needed
% - feel free to do whatever you want inside comments
%
% We know that TeX is extremely powerful, but give a stupid parser
% at least a chance to read your patterns.
%
% For more information see
%
%    http://tug.org/tex-hyphen
%
%------------------------------------------------------------------
%
%%%%%%%%%%%%%%%%%%%%%%%%%%%%%%%%%%%%%%%%%%%%%%%%%%%%%%%%%%%%%%%
% The Portuguese TeX hyphenation table.
% (C) 2015 by  Pedro J. de Rezende (rezende@ic.unicamp.br)
%          and J.Joao Dias Almeida (jj@di.uminho.pt)
% Version: 1.3 Release date: 12/08/2015
%
% (C) 1996 by  Pedro J. de Rezende (rezende@ic.unicamp.br)
%          and J.Joao Dias Almeida (jj@di.uminho.pt)
% Version: 1.2 Release date: 07/21/1996
%
% (C) 1994 by Pedro J. de Rezende (rezende@ic.unicamp.br)
% Version: 1.1 Release date: 04/12/1994
%
% (C) 1987 by Pedro J. de Rezende
% Version: 1.0 Release date: 02/13/1987
%
% -----------------------------------------------------------------
% Remember! If you *must* change it, then call the resulting file
% something  else and attach your name to your *documented* changes.
% =================================================================
%
\patterns{
1b2l
1b2r
1ba
1be
1bi
1bo
1bu
1bá
1bâ
1bã
1bé
1bí
1bó
1bú
1bê
1bõ
1c2h
1c2l
1c2r
1ca
1ce
1ci
1co
1cu
1cá
1câ
1cã
1cé
1cí
1có
1cú
1cê
1cõ
1ça
1çe
1çi
1ço
1çu
1çá
1çâ
1çã
1çé
1çí
1çó
1çú
1çê
1çõ
1d2l
1d2r
1da
1de
1di
1do
1du
1dá
1dâ
1dã
1dé
1dí
1dó
1dú
1dê
1dõ
1f2l
1f2r
1fa
1fe
1fi
1fo
1fu
1fá
1fâ
1fã
1fé
1fí
1fó
1fú
1fê
1fõ
1g2l
1g2r
1ga
1ge
1gi
1go
1gu
1gu4a
1gu4e
1gu4i
1gu4o
1gá
1gâ
1gã
1gé
1gí
1gó
1gú
1gê
1gõ
1ja
1je
1ji
1jo
1ju
1já
1jâ
1jã
1jé
1jí
1jó
1jú
1jê
1jõ
1k2l
1k2r
1ka
1ke
1ki
1ko
1ku
1ká
1kâ
1kã
1ké
1kí
1kó
1kú
1kê
1kõ
1l2h
1la
1le
1li
1lo
1lu
1lá
1lâ
1lã
1lé
1lí
1ló
1lú
1lê
1lõ
1ma
1me
1mi
1mo
1mu
1má
1mâ
1mã
1mé
1mí
1mó
1mú
1mê
1mõ
1n2h
1na
1ne
1ni
1no
1nu
1ná
1nâ
1nã
1né
1ní
1nó
1nú
1nê
1nõ
1p2l
1p2r
1pa
1pe
1pi
1po
1pu
1pá
1pâ
1pã
1pé
1pí
1pó
1pú
1pê
1põ
1qu4a
1qu4e
1qu4i
1qu4o
1ra
1re
1ri
1ro
1ru
1rá
1râ
1rã
1ré
1rí
1ró
1rú
1rê
1rõ
1sa
1se
1si
1so
1su
1sá
1sâ
1sã
1sé
1sí
1só
1sú
1sê
1sõ
1t2l
1t2r
1ta
1te
1ti
1to
1tu
1tá
1tâ
1tã
1té
1tí
1tó
1tú
1tê
1tõ
1v2l
1v2r
1va
1ve
1vi
1vo
1vu
1vá
1vâ
1vã
1vé
1ví
1vó
1vú
1vê
1võ
1w2l
1w2r
1xa
1xe
1xi
1xo
1xu
1xá
1xâ
1xã
1xé
1xí
1xó
1xú
1xê
1xõ
1za
1ze
1zi
1zo
1zu
1zá
1zâ
1zã
1zé
1zí
1zó
1zú
1zê
1zõ
a3a
a3e
a3o
c3c
e3a
e3e
e3o
i3a
i3e
i3i
i3o
i3â
i3ê
i3ô
o3a
o3e
o3o
r3r
s3s
u3a
u3e
u3o
u3u
1-
}
\hyphenation{% Do NOT make any alterations to this list! --- PdR
hard-ware
soft-ware
}
