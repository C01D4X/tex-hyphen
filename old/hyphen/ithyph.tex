%%%%%%%%%%%%%%%%%%%%%%%%%%%  file ithyph.tex  %%%%%%%%%%%%%%%%%%%%%%%%%%%%%
%
% Prepared by Claudio Beccari   e-mail  claudio.beccari@polito.it
%
%                                       Dipartimento di Elettronica
%                                       Politecnico di Torino
%                                       Corso Duca degli Abruzzi, 24
%                                       10129 TORINO
%
% Copyright  1998, 2008 Claudio Beccari
%
% This program is free software; it can be redistributed and/or modified 
% under the terms of the GNU Lesser General Public Licence,
% as published by the Free Software Foundation, either version 2.1 of the
% Licence or (at your option) any later version.
%
% \versionnumber{4.8g}   \versiondate{2008/03/08}
%
% These hyphenation patterns for the Italian language are supposed to comply
% with the Recommendation UNI 6461 on hyphenation issued by the Italian
% Standards Institution (Ente Nazionale di Unificazione UNI).  No guarantee
% or declaration of fitness to any particular purpose is given and any
% liability is disclaimed.
%
% See comments at the end of the file after the \endinput line
%
{\lccode`\'=`\'      % Apostrophe has its own lccode so that it is treated
                     % as a letter
                     %>> 1998/04/14 inserted grouping
                     %
\patterns{
.a3p2n               % After the Garzanti dictionary: a-pnea, a-pnoi-co,...
.anti1  
.anti3m2n
.bio1
.ca4p3s
.circu2m1
.contro1
.di2s3cine
.e2x1eu
.fran2k3
.free3
.li3p2sa
.narco1
.opto1
.orto3p2
.para1
.poli3p2
.pre1
.p2s
.re1i2scr
.sha2re3
.tran2s3c 
.tran2s3d 
.tran2s3l 
.tran2s3n 
.tran2s3p 
.tran2s3r 
.tran2s3t
.su2b3lu   
.su2b3r
.wa2g3n
.wel2t1
2'2
a1ia 
a1ie  
a1io  
a1iu 
a1uo 
a1ya 
2at.
e1iu 
e2w
o1ia 
o1ie  
o1io  
o1iu
1b   
2bb   
2bc   
2bd  
2bf  
2bm  
2bn  
2bp  
2bs  
2bt  
2bv
b2l   
b2r   
2b.  
2b' 
1c   
2cb   
2cc   
2cd  
2cf  
2ck  
2cm  
2cn  
2cq  
2cs  
2ct  
2cz
2chh  
c2h   
2chb 
ch2r 
2chn 
c2l  
c2r  
2c.  
2c'
.c2
1d   
2db   
2dd   
2dg  
2dl  
2dm  
2dn  
2dp  
d2r  
2ds  
2dt  
2dv  
2dw
2d.   
2d' 
.d2
1f   
2fb   
2fg   
2ff  
2fn  
f2l  
f2r  
2fs  
2ft  
2f.  
2f' 
1g   
2gb   
2gd   
2gf  
2gg  
g2h  
g2l  
2gm  
g2n  
2gp  
g2r  
2gs  
2gt
2gv   
2gw   
2gz  
2gh2t     
2g.  
2g' 
1h   
2hb   
2hd   
2hh  
hi3p2n    
h2l  
2hm  
2hn  
2hr  
2hv  
2h.  
2h' 
1j   
2j.   
2j' 
1k   
2kg   
2kf   
k2h  
2kk  
k2l  
2km  
k2r  
2ks  
2kt  
2k.  
2k' 
1l   
2lb   
2lc   
2ld  
2l3f2     
2lg  
l2h  
2lk  
2ll  
2lm  
2ln  
2lp
2lq   
2lr   
2ls  
2lt  
2lv  
2lw  
2lz  
2l.  
2l'.
2l'' 
1m   
2mb   
2mc   
2mf  
2ml  
2mm  
2mn  
2mp  
2mq  
2mr  
2ms  
2mt  
2mv  
2mw
2m.   
2m'  
1n   
2nb   
2nc   
2nd  
2nf  
2ng  
2nk  
2nl  
2nm  
2nn  
2np  
2nq  
2nr
2ns
n2s3fer
2nt   
2nv  
2nz  
n2g3n     
2nheit   
2n.  
2n'  
1p   
2pd   
p2h   
p2l  
2pn  
3p2ne 
2pp 
p2r  
2ps  
3p2sic 
2pt  
2pz  
2p.  
2p'
1q   
2qq   
2q.   
2q'
1r   
2rb   
2rc   
2rd  
2rf  
r2h  
2rg  
2rk  
2rl  
2rm  
2rn  
2rp
2rq   
2rr   
2rs  
2rt  
r2t2s3 
2rv 
2rx  
2rw  
2rz  
2r.  
2r'
1s2  
2shm 
2sh.
2sh' 
2s3s  
s4s3m 
2s3p2n   
2stb 
2stc 
2std 
2stf 
2stg 
2stm 
2stn
2stp  
2sts  
2stt 
2stv 
2sz  
4s.  
4s'.
4s''
1t   
2tb   
2tc   
2td  
2tf  
2tg  
t2h  
t2l  
2tm  
2tn  
2tp  
t2r  
t2s
3t2sch      
2tt  
t2t3s
2tv  
2tw  
t2z  
2tzk 
tz2s 
2t.  
2t'.
2t''
1v
2vc   
v2l   
v2r  
2vv  
2v.  
2v'.
2v''
1w   
w2h   
wa2r  
2w1y 
2w.  
2w'
1x
2xb
2xc 
2xf
2xh
2xm
2xp  
2xt   
2xw   
2x.   
2x'
y1ou 
y1i
1z   
2zb   
2zd   
2zl  
2zn  
2zp  
2zt  
2zs  
2zv  
2zz  
2z.  
2z'.
2z''  
.z2
}}                          % Pattern end

\endinput


%%%%%%%%%%%%%%%%%%%%%%%%%%%%%%% Information %%%%%%%%%%%%%%%%%%%%%%%%%%%%%%%

As the previous versions, this new set  of  patterns does  not  contain  any
accented  character  so  that  the hyphenation algorithm behaves properly in
both cases, that is with OT1 and T1 encodings.   With  the  former  encoding
fonts  do  not contain  accented characters,  while with the latter accented
characters are present and sequences such as � map directly to slot "E0 that
contains "agrave".

Of course if you use T1 encoded  fonts you get the full power of the hyphen-
ation algorithm, while if you use OT1 encoded  fonts you  miss some possible  
break  points;  this  is  not a big inconvenience in Italian because:

1) The Regulation UNI 6015 on  accents  specifies  that  compulsory  accents
   appear  only  on the ending vowel of oxitone words (parole tronche); this
   means that it is almost  indifferent  to have or to miss  the T1  encoded 
   fonts because the only difference consists in how TeX  evaluates  the end 
   of the word;  in practice  if you have  these special  facilities you get 
   "qua-li-t�", while   if  you miss them, you get "qua-lit�" (assuming that
   \righthyphenmin > 1).

2)  Optional  accents are so rare in Italian, that if you absolutely want to
   use  them  in  those  rare  instances,  and  you  miss  the  T1  encoding
   facilities, you should also provide  explicit discretionary hyphens as in
   "s�\-gui\-to".

There is no explicit  hyphenation  exception  list  because  these  patterns
proved  to  hyphenate correctly a very large set of words suitably chosen in
order to test them in the most heavy circumstances; these patterns were used
in the preparation of a number of books and no errors were discovered.

Nevertheless if you frequently use  technical terms that you want hyphenated
differently  from  what  is  normally  done  (for  example  if  you   prefer
etymological  hyphenation  of  prefixed  and/or  suffixed  words) you should
insert a specific hyphenation  list  in  the  preamble of your document, for
example:

\hyphenation{su-per-in-dut-to-re su-per-in-dut-to-ri}

If you use, as you should, the italan  option of the babel package, then you 
have available the active charater "  that allows you to put a discretionary 
break at a word boundary of a compound word while maintaning the hyphenation 
algorithm on the rest of the word. 

Please, read the babel package documentation.

Should you find any word that gets hyphenated in a wrong way, please, AFTER
CHECKING ON A RELIABLE MODERN DICTIONARY, report to the author, preferably
by e-mail.


                       Happy multilingual typesetting!
